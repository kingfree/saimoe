\album{咲-Saki- THE~夢のヒットスクエア~キャラソン対局編}

\songTitled{麻雀天使にかこまれちゃう}{\\
宮永咲(植田佳奈)、天江衣(福原香織)、池田華菜(森永理科)、加治木ゆみ(小林ゆう)}
{\albumName}
\songMemo{作詞:\B{畑~亜貴}/作曲・編曲:\B{木下智哉}}
\songText{\kasho
わかる わかる\\
とっておきのREVOLUTION\\
キミもどうだい?\\

なんと今日の挑戦は\\
初めましてなんかいらない\\
全部いただきまーす\\

遠慮だってついに消えちゃうよ\\
だからもっと運の真髄と\\
そうだ 戯れよう?\\

この世に嵐 ここだけ雷雨\\
ワクワクしてきた 大興奮!\\

時間は当然流れるけど\\
ねー?\\
関係なくなるから\\
快感体験まだでしょ\\
キミは キミは やってみたいはず\\

天和なんてやってみたいし\\
緑一色は美しいし\\
九蓮したって 死んじゃわないよ\\
配牌だけじゃない\\

毎回違うセカイ\\
楽しもうよ ネギしょって来い\\

よいね天の混乱が\\
招きますよちょいと地獄\\
決心ついたかなー\\

平和だったそうね大体ね\\
ここにあった牌の存在は\\
恵みか 罠か?\\

握る幸せ 捨てて絶望\\
ゾクゾクするよね 熱中戦!\\

終わりは絶対ゆずれないよ\\
もー!\\
同情する余地なし\\
なーい!\\
完全完璧だからね\\
つぎは つぎは こっちに立つはず\\

地和なんてやったことあるし\\
大三元は よくあるテンパイ\\
清老だって できそうだよ\\
自摸牌だけじゃない\\

対局それがミライ\\
面白いよ ダシとって来い\\

時間は当然流れるけど\\
ねー?\\
関係なくなるから\\
おーい!\\
快感体験まだでしょ\\
キミは キミは やってみたいはず\\

人和だってやったことあるし\\
小四喜は よくあるテンパイ\\
字一色も合わせたいよ\\
配牌だけじゃない\\

毎回違うセカイ\\
楽しもうよ ネギしょって来い
}

\songTitled{予感、咲きました!}{宮永咲(植田佳奈)}
{\albumName}
\songMemo{作詞:\B{畑~亜貴}/作曲・編曲:\B{村井~大}}
\songText{\kasho
探せないな迷子の\\
素直な気持ちはどこ?\\
あの頃から逃げたのかもしれない\\

しかたないね小さな\\
心のなかでひとり\\
遊ばないって決めてたよ\\

それでも輝いて\\
進むみんながいた\\
私だって 私だって\\
夢に会いたい\\

想いが開くとき\\
熱い奇跡の花\\
四つの花びら舞い降りて来るよ\\
指先に予感で\\
踊る 今の今が\\
私を待ってたの? この場所が謎の欠片\\

見つけたいと本気で\\
祈れば悲しみから\\
浮かび上がる怖くなって泣きそう\\

むずかしいね過去から\\
思い出目覚めなさい\\
勇気もっと咲かせるよ\\

臆病な眼差しじゃ\\
先に行けないから\\
強くなって 強くなって\\
夢を飛びたい\\

いつでも偶然と\\
言われるだけじゃなく\\
四つが一つの光呼ぶ景色\\
描いてみせるのが\\
不思議 誰と誰を\\
私は待ってるの? この場所で決めてしまおう!\\

想いが開くとき\\
熱い奇跡の花\\
四つの花びら舞い降りて来るよ\\
指先に予感で\\
踊る 今の今が\\
私を待ってたの? この場所にいると決めた!
}

\songTitled{刹那の海よ}{天江衣(福原香織)}
{\albumName}
\songMemo{作詞:\B{畑~亜貴}/作曲・編曲:\B{村井~大}}
\songText{\kasho
海底 海底\\
此の場 崩壊 崩壊 闇の真価\\
流局 流局\\
何れ月光参じて 刹那歓喜\\
西入 西入\\
誰ぜ神の路 覗くまいて\\
降臨 降臨 俗の世へと\\

9600 闇聴 怯えの手\\
槓裏 マルノリ 救い降る\\
海底撈月 見て居れよ\\
さぞ さぞ 苦しかろう\\

幽谷に\\
ひとりぼっち 退屈があるのみ\\
ひとりぼっち 戯れに興じて\\
身を 身を 滅ぼすまで\\
悟り得ぬ定めに\\
ひとりぼっち 仮初めの友でも\\
ひとりぼっち 愚かしい笑みでも\\
気が 気が 浮かれるまま\\
人に交わりて 暫し留まれり\\

即リー 即リー\\
彼の地 生還 生還 希望皆無\\
連荘 連荘\\
誉れ 同時に奉ずる 刹那頂点\\
二翻縛り 八連荘\\
誰ぞ鬼と化し とどめ刺して\\
茫然 茫然 暗き永遠よ\\

現物 ベタオリ 最果ての\\
八連しそうな 光など\\
オーラスリーチも 届かぬよ\\
あな あな 口惜しかろう\\

単簡な\\
あそびあいて 漁火焚きあげろ\\
あそびあいて 呼ぶ業の易きよ\\
目と 目が 惹かれ良かれ\\
言の葉でいざなう\\
あそびあいて 待ち続けた日々は\\
あそびあいて おののく者ばかり\\
其は 其は 見つけたのか\\
人は面白き 波を起こすなり\\

18000 直撃 震える手\\
表裏 マルノリ もう一度\\
海底撈月 汝等には\\
さて さて 会うべきだろう\\

有終の\\
ひとりぼっち 退屈がまさかの\\
ひとりぼっち 戯れに変じて\\
身を 身を 預けるなら\\
祭りへと急ぎて\\
あそびあいて 待ち続けた日々よ\\
あそびあいて 別れを告げるべき\\
気が 気が 浮かれるまま\\
人に交わりて 人は友を得る
}

\songTitled{イキナリナリユキナリッ}{池田華菜(森永理科)}
{\albumName}
\songMemo{作詞:\B{畑~亜貴}/作曲・編曲:\B{tetsu-yaeh}}
\songText{\kasho
「楽しんで」って言われたよ\\
そのコトバと 歩いてゆこッ\\
前向いて 上を向いて\\
ナリユキ次第やるんだ\\

あれ? あれれ? 字牌ばかり\\
あれ? あれれ? 切るとく〜る\\

まわりが魔物なら\\
あたしは退魔師かいッ\\
キリキリ舞々 ありゃなんなんだー!\\
ちょーし出ない\\

まわりの空気まで\\
あたしを迷わせた\\
ムリムリ精々 ハッタリかまして\\
しがみつくぞ\\

まーたヒドイ あぶないよあぶない\\
自分らしさ? あーッ テンション落ちるよ\\

「楽しんで」「へコまないで」\\
そのコトバを もいちど言って\\
前向いた 上を向いた\\
イキナリきたよキツイッ\\
「和了れない」「さ、どーするのさ?」\\
アタマがゲンカイ アイツらサイテー\\
テンパるも 役がなくて\\
イキナリ今日はつらいな\\

あれ? あれれ? ちーちゃばかり\\
あれ? あれれ? じごなのね\\
あれ? あれれ? 字牌ばかり\\
あれ? あれれ? 切るとく〜る\\

くやしい動けない\\
だれかが魔術師かいッ\\
ジワジワ続々 こりゃおかしいよー!\\
チカラ抜ける\\

くやしく終わるのかッ\\
だれかに呪われた?\\
ヤミヤミ津々 サッパリわかんねぇ〜\\
みんなごめん\\

うーわイカン まけそうだまけそう\\
降参しない! ねーッ チャンスつかんで\\

「楽しめない」「悪い夢だ」\\
ふるえる足 あたしの抵抗\\
ナミダには 早いよほら\\
ナリユキ次第やるって!\\
「邪魔すんな」「じゃ、どーなるのさ?」\\
ココロがホーカイ くるしージョウキョウ\\
あたり牌 しょせん数枚\\
ナリユキ次第だめかな\\

「楽しんで」「へコまないで」\\
そのコトバを もいちど言って\\
前向いた 上を向いた\\
イキナリきたよキツイッ\\
「和了れない」「さ、どーするのさ?」\\
アタマがゲンカイ アイツらサイテー\\
結局は ひとりノーテン\\
ヤッパリ今日はつらいな\\

あれ? あれれ? ちーちゃばかり\\
あれ? あれれ? じごなのね\\
あれ? あれれ? 字牌ばかり\\
あれ? あれれ? 切るとく〜る
}

\songTitled{見えない君の探し方}{加治木ゆみ(小林ゆう)}
{\albumName}
\songMemo{作詞:\B{畑~亜貴}/作曲:\B{田代智一}/編曲:\B{安岡洋一郎}}
\songText{\kasho
ここまで やっとここまで\\
来たもちろん 簡単じゃない\\
あの頃 探せど探せど見えない君\\

戦いを共にしてみたいオーラ\\
受け取るこころが騒ぎ出す\\
扉を開ける 答えを持たず\\
思わず叫んだよ\\

聴牌なら君も\\
ゆらり出る現る\\

誰でも きっと誰でも\\
武器はまだ 隠しておくさ\\
鍛えろ そっと鍛えろ\\
武器になれ 秘密のちから\\
ここまで やっとここまで\\
来たもちろん 簡単じゃない\\
ヒキとか ツキとか越えたい\\
こころを動かす駆け引き…\\
\newcolumn
仲間さえ忘れ特別な能力\\
磨いてますます溶け込んで\\
孤独は味方 自分を守れ\\
必ず勝てるから\\

配牌から希望\\
さらり受け赴く\\

探せど ずっと探せど\\
気配つかむ 手がかりは無し\\
あの頃 知ったあの頃\\
決めていた ちからが欲しい\\
ここまでやっとここまで\\
来たもちろん 偶然じゃない\\
鳴きとか 槓とかしないで\\
こころを覗けば魔物が…\\

誰でも きっと誰でも\\
武器がまだ 眠っているさ\\
鍛える そっと鍛える\\
武器になれ 秘密のちから\\
私は もっと君への\\
驚きを 伝えるために\\
あの頃 探せど探せど\\
見えない 君へと 叫んだ
}

\songTitled{Angel zone}{原村和(小清水亜美)}
{\albumName}
\songMemo{作詞:\B{畑~亜貴}/作曲:\B{福本公四郎}/編曲:\B{安藤高弘}}
\songText{\kasho
夢と現の境目を\\
くぐり抜ける 私が見えた\\

冴えて目覚めて火照り出す\\
軽くなる身体は\\

Angel zone 越えてる\\
もう誰もいないの\\
ふわっと ふわっと 大きいね\\
光った雲に Touch\&kiss\\

決めましょう 迷彩で\\
行きましょう ひっかけ\\
あなたといつも一緒にいたい\\
決まりです! ダマテン\\
行くべきです 直\\
絆の糸(切れない糸)感じてますよね?\\
My soul friend\\
\newcolumn
地上も天もそこにある\\
望むたびに 浮かび上がるの\\

怒る悲しむ抱きしめる\\
隠せない気持ちで\\

Angel style 遠くで\\
あの輝きは何?\\
わかった わかった やっぱりね\\
飛ばした合図 Touch\&catch\\

好きでしょう 危険牌\\
燃えましょう 三家和\\
あなたはもっとしっかり咲くわ\\
好きこそっ! おっかけ\\
燃え足りない オーラス\\
明日はまだ(これからなの)信じてますから!\\
I hope all\\

行きましょう 親リーで\\
決めましょう リーソク\\
あなたといつも一緒にいたい\\
決まりです! 壁だから\\
行くべきです とおし\\
絆の糸(切れない糸)感じてますよね?\\
My soul friend
}

\songTitled{逃しません…ですわ!}{龍門渕透華(茅原実里)}
{\albumName}
\songMemo{作詞:\B{畑~亜貴}/作曲:\B{田代智一}/編曲:\B{近藤昭雄}}
\songText{\kasho
目覚めなさい 目覚めなさい\\
はじめなさい はじめなさい…ですわ!\\

そうね わたくしになってみたい?\\
勘違い まぁいいわよ アイドルの悩み\\

常にヒロイン 勝利しか\\
味わいたくありませんの\\

強い方 求めますわじっくり\\
あーら 貴女たちいかが?\\
冷静な指で 確かめたくなる\\
興奮と刺激のステージ\\

思惑が外れたらピクピクと\\
つい短気な癖で\\
入れば リーチですーわ\\
おふざけもたいがいに\\
あ〜〜 しくされですわーッ!\\
デジタルモード 無視\\
わたくしは賭ける まだまだ\\

目覚めなさい 目覚めなさい\\
挑みなさい 挑みなさい…ですわ!\\

なあに わたくしはデンジャラス?\\
噂では さぁどうなの 目立つのね美貌が\\

立てばミネルヴァ 勝負では\\
華麗に打つ乙女ですの\\

怖い役 素敵ですわうっとり\\
ねーえ お楽しみいかが?\\
欲望に駆られ 余裕のつもりが\\
オーラスで主役はトラブル\\

プライドを潰せるわゾクゾクと\\
ほら夢の終わりが\\
親倍 8000オール\\
だれよりもなによりも\\
あ〜〜 目立ちますわーッ!\\
デジタル対決 無我\\
わたくしのために うてうて\\

強烈な爽快感でワクワクと\\
盛り上がる想いは\\
直撃 狙い撃ちですわ\\
おふざけもたいがいに\\
あ〜〜 しくされですわーッ!\\
デジタルモード 無視\\
わたくしは賭ける まだまだ\\

逃がしません 逃がしません\\
はなしません はなしません…ですわ!\\

お首洗いお待ちなさーい!!
}

\songTitled{ひとりにひとつ}{福路美穂子(堀江由衣)}
{\albumName}
\songMemo{作詞:\B{畑~亜貴}/作曲・編曲:\B{前田克樹}}
\songText{\kasho
つらかった事など忘れましょう\\
次からは気分を変えて がんばれる\\

だってね ひとりにひとつ\\
ハッピーなちからをくれる\\
運命があなたを守るの\\

トイツの ようです\\
優しい夢の お手伝いするわ\\
飛び立つまでは みな危うい\\
小鳥でいいのよ\\
すこしふらり 自分の道へと\\
続く場所探して 対々和ドラ3\\

出会いって切ない瞬間でしょう\\
繰り返しこころに風を おいてゆく\\

まってて ひとりがふたり\\
ラッキーなつながりはこぶ\\
毎日あなたを見てるわ\\

染め手の ようです\\
とめちゃだめよ 流れない水は\\
濁ってしまうの さあここから\\
旅立つときです\\
ゆれる枝が 自由を導く\\
大丈夫いつでも 面混白中\\

チャンタの ようです\\
優しい夢の お手伝いするわ\\
飛び立つやがて あの大空\\
目指してきれいな\\
軌道えがく 自分の道へと\\
続く場所探して 純全三色
}

\songTitled{ステルス・モ・モ・モード}{東横桃子(斎藤桃子)}
{\albumName}
\songMemo{作詞:\B{畑~亜貴}/作曲・編曲:\B{村井~大}}
\songText{\kasho
どこでも静かに消えれば\\
存在未満のまぼろし\\
私はすべてを見渡す\\
音も声も届かない\\

呼ばれて気付いた\\
自分を求める人が嬉しいっすよ\\

たぶん何もいらなかった\\
でも今はがんばるっす\\
ただついてくだけじゃなくて\\
想いをぶつける\\
次は! そばで待ってるよ\\
次は! ダマと同じリーチ\\
マジに 消えます\\
ステルスモモで 空気より軽い\\

いさかい起きてもいないし\\
観察次第のなりゆき\\
私の能力ムダかも\\
どうせ奇跡起こらない\\

大きく激しく\\
チカラを認めて探してくれたっすよ\\

他人なんていらなかった\\
もしあの日あの場所で\\
またすがたをくらませても\\
きっと降参っすよ\\
隣り! 多分ツモがいいね\\
隣り! 待ちもキレイきっと\\
では 消えましょう\\
マイナス気配 透明の彼方へ\\

たぶん何もいらなかった\\
でも今はがんばるっす\\
ただついてくだけじゃなくて\\
想いをぶつける\\
次で! ドラが重なれば\\
次で! 高めひいてツモ\\
マジに 消えます\\
ステルスモモは 空気さえ自在\\

どこでも静かに消えれば\\
存在未満のまぼろし\\
私はそろそろ沈むよ\\
独りきりが見抜けない\\
音も声も届かない
}
