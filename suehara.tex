\ctexset{
section/name = {第,局},
section/number = \chinese{section}
}

\part{咲 -Saki- 萌战篇}
\setcounter{section}{0}

末原@日萌2014麻将厨团

挖掘机技术哪家强?

其实,这并不仅仅是一句过气儿的蓝翔梗而已。

正负零的结局,怕是没有比这更棒的了。麻将厨群的故事无论如何都是一个非常励志的故事,有时候生活就是比剧本更精彩。由于总结的定位,基本上以反应真实面貌为主;或许增加一些艺术化加工的话,能成为一个不错的剧本呢。

感谢津津大厨为我写序。

\section*{序}
\addcontentsline{toc}{section}{序}

津津@日萌2014麻将厨团

军师找我写序的时候,我感到受宠若惊。作为一个现充学生党,萌战后期经常不见人影,能做的也就是平时(在群里)编梗吹水,战时(给大厨)揉肩捶腿。

这里的大厨数量基本不超过4,萌战本来就是十几个人的游戏,当然,其中给麻将厨团带来命运转折的,就是军师了。

关于军师其人,现在都还是个谜,他在总结中还写到为了萌战特意注册了qq小号,因此我们也只能脑补。目前只知道他的本行就是写代码,一手开发了麻将厨团的计票器和挖掘机。由于军师本人对技术的精益求精,后来还陆续出现了挖掘机ver.2,ver.3以及最终的超级无敌豪华加强版钛合金挖掘机。开场人手一台即便是半自动的挖掘机,和预选时vg刷票的恶心无力相比,也感觉爽的飞起。再次感叹技术是第一生产力,落后就要挨打。

军师很早就提到要写萌战总结,我们都觉得这个主意好,给自己一份纪念,给支持者一碗鸡汤,给围观者一个真相,给好事者一记耳光。

于是就有了下面这个故事。从本战到决赛共24局,从标题到标点均出自军师之手。没有惊才绝艳的文笔,都是一步一步走来踏踏实实的印记。这份总结的字里行间满溢着一位大厨最真挚的爱意,我相信每一个爱着自己本命的人都会明白。

这是个极其励志的故事,引用军师的一段原文:“想想当年一群新厨子、票力30票的弱鸡麻将,二回战兔子海爷隔空对战干翻了邪教,三回战神鸭600+打赢了电磁,FL阶段大魔王第一次取得了对圆胜利;再开后,借着没有服务器的东风,拿下了粉、黄,靠着圆麻协议淘汰了黄图;这次,终于连世界的主宰者——运营——这个“不可能战胜”的敌人都被打到放弃治疗了!如此励志而逆天的故事,不由得泪流满面——真的,有时候就是这样,生活中实实在在发生的事情,比那些少年漫画还精彩。”

最后说下自己的一点感想。当初被安利进群的时候还是6月初的度萌,那时群里只有7个人,没有想到会厨日萌,更没有想过会拿萌王。回想刚刚起步的预选,聊天中出现最多的一个词是【战五】,并伴随着各种心塞自嘲的表情,那个时候还只是以塞人进本战一轮游为目标。

后来,慢慢地,大家都来了,不知不觉半年过去,今晚就是咲和会师决赛的时刻。我在敲字的时候,军师还在群里吼自由出票面,大厨还在勤恳干活值班,日语菊苣还在努力憋萌文,水军还在吹水编段子。群里的气氛一如往常欢乐祥和,没有即将加冕的狂喜,只有圆满结束的欣慰与满足。

半年的风雨即将尘埃落定,咲–Saki– 萌战篇,与君共勉。



\section{萌动}
空有麻厨无大志,只靠圆力且偷生

      个人票力\footnote{注:个人票力不计服务器,厨团合计不计日租;定义为拿到的CODE,并非指出票。}:1 厨团合计:30

说实话,我很多年前就关注萌战了;一直也断断续续或能拿到1票,或被规制。近两年,靠着一个日本的VPS,也算是稳定的一般参加者了。本来以为今年也会继续做这样的看客,却被一封贴吧的私信打破了。这份平静。

 “和我签订契约,成为麻将大厨吧!”

咦,不是成为魔法少女厨小圆脸吗。无论如何,注册了一个新的QQ,像好多故事的开头一样,被莫名其妙卷入了另一个世界——日萌,大厨的世界。那是七月底,天气仍然燥热的季节。

加群以后,果然还是逃不掉圆脸的关系。“争取16强”“8强不定”“不进决赛”“看圆脸脸色”,还有鬼扯的“钉宫百胜”。我都想退群来着。想当年8强占6的麻将军团,如今已经落得如此下场?“其实还是实力不济”“没有票啊”“群里有圆厨”——唉。面对满屏幕这样的回答,只能一声叹息。不过,也算谦虚地表达了自己的态度:工作党,有技术,我也不知道能投入多少精力;但是既然现在暂时选择参与,那么我希望我喜爱的角色,至少是支持的作品,能走到最后。另外,乳和还是精神领袖,倍感欣慰。

说来被召唤,大概还是因为计票器。其实那个东西也蛮简单,一开始只是个想法,然后也还是为了方便自己做个“看客”;谁知就这么做下来了,从计票器开始,渐渐地深入到日萌更多更复杂的技术世界中去。最开始的时候计票器在厨群公开,逐步增加功能,甚至被当作核心技术,一度被自由反对放到外面;不过呢,计票器这种东西,独乐乐不如众乐乐,不如带给更多看客更多方便。于是我还是发布了,而且GPL开源了:希望更多人,更多懂技术的人,能一起来玩儿。


\section{弃保}
东风不与卷饼便,天降软妹杀二麻

      个人票力:5 厨团合计:60

预选前十占十的丰功伟绩已经达成,顶着洪德法也将那43人的入围本战记录刷新到44人。麻厨还沉浸在胜利的喜悦中,本战正式开始了——那时,我还只是个看客而已。

果然卷饼自带“东风”外挂,附带“起家”效果,本战第一天就出场呢。可惜的是,本团领袖自由同学决定弃卷饼保画板,甚至连他的“钉宫百胜”都抛在脑后了。但是逐渐看着形势不对,“未知”的势力在刷卷饼。直到下午,掉头刷卷饼——为时晚矣!最终被天降软妹的Beta双杀。

开局第一场,东风不利。倒是也蛮符合原作的。不过这场比赛输了,死了一个卷饼是小事,对信心和弃保方针的打击还是巨大的。原本确立的全国>本篇>阿知贺的弃保策略受到了严重的打击,加上开学影响,大厨难以忘怀的温暖(下称暖姐,与松实宥不同)等由于开学无暇刷票,麻厨人手紧张得很。除领袖自由外,大概只有津津、银狼等很少的几个人能继续刷票了。不过反过来看,也发现了我们之外的麻将势力的存在,盼着得到“未知”的支援——这就好比打牌时自己穷光蛋盼着队友一手好牌的心态吧。

至于个人票力方面,挖掘了一下能用的VPN,个人总算是勉强达到“日常5票”的厨团最低标准了。我还在熟悉萌战的技术中,当时仍然是忐忑不安的心态——别给大家拖后腿丫。拿票的事情先不说,找来了发行所代码认真研读一下,事后证明这是非常重要的。如果还有来年的话,也建议以后的技术厨都要先学习一遍发行所代码。不过经过这一战,发现果然只有几十票是远远不够的。各种苦逼、各种不顺、各种无力、各种悲观,24强的剧本也变成只剩7个甚至4个麻将的名额了。


\section{险胜}
目录折戟撞成渣,爱姐出马一票杀

      个人票力:5 厨团合计:80

好在一回战火药味还是不是那么浓烈,大多只是些平淡的比赛而已。这也正好给了大家一个熟悉萌战的好时机,毕竟没有比赛的话好多事情都是没法做的。

这时候梨落作为“去年的麻厨”还在内群,据说自由也在对面邪教的内群。梨落一直再给我们讲着“过去的故事”,嗯,大概是吧。总之,从他讲的内容来看,前人们玩儿萌战的时候都好高大上,各种花销五位数,各种基情撕逼,开多台服务器“多线程”拿票,我们一群屌丝也只能望洋兴叹了。正想着怎么办的时候,第一次和其他大阵营的正面冲突来了——爱姐vs目录。

说实话,这一场拿下真是靠运气。当天开始上了3台服务器,同时梨落和自由还处于蜜月期,我们得到了来自梨落的大力支持。圆脸这边也能提供一些票力,还有那“未知”的战友——同时,对面出票一直在撞撞撞IP,在我的计票器里尤其明显的一大片绿色(后改为蓝色)ID。最后算下来,爱姐一票险胜;而数数因为撞IP而死掉的票的话,爱姐死掉8票而目录死掉24票,真是——太惊险了。

无论如何,即使是靠着运气的1票险胜,还是让麻厨士气大涨。不过更涨士气的,大概还要算稍后的技术进步。日萌的验证码,虽然我也研究过,但是自己的程序一直效果不好,同时也不敢确定自己的做法是否合适;直到在梨落的搭桥下和零件(灵剑2006)直接交流了一下,确认了方案的可行,并最终上线。此外,还咨询了计算票尾偏离的算法,这方面虽然我后来自己考虑过其他算法,但最终还是使用了零件的算法,并在11月初的2014.0.8版计票器中上线,在此表示感谢。


\section{分合}
四周六日天天咲,两死一生夜夜哭

      个人票力:5 厨团合计:200

大战后,“未知”的战友渐渐浮出了水面——红枫叶大厨加入了我们,单人每天接近三位数的票力当时我们都震惊了!外加每天凌晨不睡觉,简直是神。不过大家聊聊原来第一场玩儿命刷卷饼的就是枫叶——未知的战友啊,相见恨晚啊!各种意义上。总之,随着有经验的大厨的加入,再加上服务器,整体票力强大了很多,大家的士气也比之前好了很多,比赛也算是一路顺风顺水。

同时,在这段时间,技术也逐步发展成熟。服务器的产票率逐渐提升到了每台每天20票以上。另外,自动出票的程序也总算是上线了。按理说写自动出票这种程序一点难度都没有,只是Windows看起来用起来不爽而已——哎我都在纠结什么呢。

后来突如其来的84规制倒是也没什么,因为读过发行所源代码,马上就知道84是端口扫描,一抓包他在扫哪些端口都一览无余,统统干掉即可——只是由于扫描了PPTP的默认端口,这VPN拿票基本是报废了。不过VG很快填补了上来,乳和本战首胜也算轻松。正当大家以为萌战都按着剧本来的时候,真姬那场开始彻底改变了。至少我这样认为。

真姬凌晨被A,下午垂直反超,同时带了隔壁的雪菜淘汰了NGNL的白。可能是当时我太活跃了吧,在战吧不知天高地厚地@ 了各家领袖。凌晨的A我们也不知道谁干的,反正下午的垂直梨落承认了;于是有了今年这一次撕逼:我在反对梨落这种不顾观赏性的垂直超多重。运营的目标应该是举办一届精彩至少是看起来精彩的比赛,而不是满目疮痍的多重大乱斗,这样完全不给运营面子嘛,是不是?而且当时心里更担心的则是——如果运营开审议,我们怎么办?


\section{乱入}
一骑绝尘战马啸,八面威风众人惊

      个人票力:5 厨团合计:200+

真姬垂直的烟云还没有散去,运营方面和2ch也都在讨论着多重对策。毕竟,前两天的真姬,赛中大家还都以为“仅仅是伪票而已”。其实,如果无法分辨真伪的话,其实比赛也还是有一定的精彩程度的。于是乎到了这一天,一个大胆的作战计划——垂直伪马!一下子,萌战的热情就这样被点 “燃”了!

2ch一片惊呼“马好强大”,众人想着前些天真姬的比赛,普遍倾向于这些是真票。同时,除了我们的伪票以外,也能看到批量的貌似真票的涌入。不仅如此,据说圆脸方面的表现:甚至有人主张用真票砸伪票,果然霸气侧漏!于是赶紧向圆脸方面汇报了我们的行动,可别闹太嗨啊。运营方面的动作则是在投票版规制了部分涉嫌“多重”的IP,其实大部分也就是我们出票用的服务器IP而已。审议之声也是此起彼伏——嘿嘿,等晚上结算的时候给你们好看!

果然,魔王、学姐的票面也逐渐追上并最终实现了反超,毫无压力地领先不少。开票的时候众人大惊失色,马的垂直爆票居然是伪票!伪票屋大战多重屋,萌战一下子又“燃”起来了!而“马”,也成为了今年萌战一大明星,更是人们长久的话题。

回头想来,这一场马的乱入,简直是神来之笔。将计就计,借力真姬垂直超多重,伪票效果得到了显著放大;同时,赛后也让大家明白,有伪票的比赛是多么的欢乐——你们愿意用那不透明的审议,去排除这种欢乐吗?此外,本土更有马厨就是麻厨的说法,进一步增强了马和麻的话题性,让“麻将票多”不再遭到前一年那样“圆脸才有人气”的非议。

不过外边和梨落的嘴炮还没结束,梨落也愤然退群。说真话我真的没想明白梨落为啥那么火大,如果我赔礼道歉还不行的话,我是不是该从萌战退场了?


\section{歼灭}
麻邪本是同根生,隔空对战何太急

      个人票力:150 厨团合计:400+

哎,和梨落吵架,退场的话实在是有点,不过还是先躲起来吧。吵架有什么用呢?又不能提升票力,又不能赢得比赛!被人骂了一顿,被人指责弱鸡、票力不足,被人叫嚣“要出150票的话,需要10台服务器”的时候——最棒的手段不就是抓起一大摞票摔对面脸上吗?可能一群人以为我不玩儿了的时候,我正在进行着我的研究——研发代号“挖掘机”。首先是VG的语言问题,我用了某种技术手段绕过了VG的这个限制,实现了任意语言的断线重连再拿票。有了这个功能,厨群的票力肯定能有一定程度的提升。最终,当我带着研发成功的“挖掘机”强势回归的时候,厨群的战友们都惊呆了。

到了9月29日。为了保证下一轮AKO不被反连记,集体讨论的结果是这次选择直接隔空A死海爷。当然邪教不会坐以待毙,死连琴里,形成了兔子-雪乃对琴里-海爷的连记大战。当然苍殿在旁边坐收渔利,并最终拿下了500+的二回战第一高票。

结果就像大家都知道的,在多方面的支援下,在挖掘机的轰鸣中,兔子和雪乃全天连记领先,没有给过邪教任何机会。即使邪教丢出了在2ch惹来一片骂声的超多重C,仍然无法获胜。其中很大的原因还是“挖掘机”,有了“挖掘机”,四大厨团中麻将的票力终于战胜了邪教,四位脱出!

说到邪教,据说都是前些年的麻厨转过去的。并且,邪教和麻将共享同一个土豪——壕流,厨笨蛋和真姬的土豪。所以说到底,本是同根生一点都不过分。于是,为什么要在这儿明着撕逼呢?或许这就是为了萌战好看吧,谁知道呢,总之,就是这么做了——甚至,梨落还扬言下轮要A 回来AKO 300票!


\section{疯狂}
运营发疯规制咲,快马加鞭向前冲

      个人票力:200 厨团合计:450+

俗话说,打是亲骂是爱。这梨落和自由隔空打了一场、嘴仗不断以后,群里渐渐有了梨自CP的呼声。同时,本着对提高比赛精彩程度的原则,我们也希望麻将的曲线更好看,确定了程控曲线为主体、适当结合人工曲线修理工的方针,并切实地研究了一些控制曲线的方法。于是乎大概从下面的比赛开始,大多数麻将的曲线基本都像韩国选美大赛一样,看起来还不错,仔细看一个样。

二回战剩余的比赛可以说基本都波澜不惊,说好的真帆让真姬也就那样让了一场,也切实地测试了曲线控制。直到最后的复活赛,事实证明二回战复活赛远远比正赛精彩。

预选几乎踩线进入二预。二预借助洪德法晋级本战。本战第一轮被伪了好多,好在最终二位。第一次复活勉强过关。这是一个什么样的角色呢?——是一匹马,哦不,是很多马!进击的巨人中的马群。话说要是一马一票的话,或许它们能拿下萌王也说不定呢。马在一回战的时候被寄托了那么多话题,2ch上一片挺马之声,甚至每天有支援有投票模板。而马本身也争气,不仅踹翻了小四喜,还顺利通过了一回复活。在二回复活,马又能有怎样的表现呢?

大家还在担心马能否晋级的时候,比赛中突然出现了一个严重的问题:运营似乎在投票版中规制了“咲”这个字。也就是说,如果票里边带有“咲”这个字,将无法在主投票版投出。这也是为何麻将军团在当天比赛后半段萎蔫的重要原因,尽管有一批票被投往了备用版。不过,看起来有不少人特别是本土参与者,在发现无法投“咲”之后怒而投马,于是马最终被刷到了二位,马上只有海爷一人——难道那么多妹纸都不如马萌吗?

这还不是最大的问题,毕竟麻厨老练,当时的自动出票程序已经可以利用备用版,只是设置上还有些小问题。真正最大的受害者是麻将?才不是呢,真正最大的受害者是伪恋里那颗苦逼的香菜——小野寺小咲同学吧。这回运营的锅可是背大了,运营直接被本土一顿海喷,信用也受到了极大的损失。

事已至此,运营也看出了光靠规制无法解决问题,而且会闹出更大麻烦。在没有CODE屋权限的情况下,审议制再度被提上日程。这回会怎样审议呢?哎,天知道,只好静观其变了。


\section{合战}
赛中无圆麻称王,深山老鸭CODE汤

      个人票力:200 厨团合计:450

3V1!麻将和别人3V1!啊?居然是三个电磁围攻一只鸭子?麻将被人多打少还真是稀奇。对面的电磁据说联合了邪教,准备一起大干一场。没辙,鸭子好歹也是圆神分身,我们也联合了圆脸。第一次火药味十足的海外四家对轰就这样开始了!

日租日租日租!VG被挖光,日租不能少。大概上了2台日租吧。经过充分准备之后,比赛就这样开始了。第一波130的发行量简直吓人,开场投票简直吓人……但是,赛前的不安,凌晨的紧张,早上的平静,下午的放松,晚上的愉悦,就这样一路走下来了。凌晨只有枫叶一个人值守,尽管有挖掘机运转,但还是票面被对面拉开很多;而随着圆脸的支援陆续到位,以及白天拿票量的增加,我们手里屯着的票逐渐多了起来——毕竟,按原著鸭子也得要到“深山”处才能发挥威力——其实是白天没人投票罢了。于是,攒足了海底,接下来为大家表演海底的大——海——啸——!美轮美奂的典型麻将式曲线。尽管待发严重,服务器这边还是基本能保障每小时1票的效率;VG呢,也受到抢线的影响,大家日子都不好过的样子。日萌也是久违了1000+的总票数,大家惊呼日萌有救!同时运营也缩了,尽管赛前宣布可能砍票,但是这一场还是没有砍票就这样结束。

最终,鸭子轰出600+,不仅胜出,还是当时的最高票。这个票数在今年,当时看是空前的;事后看,应该是绝后了吧。运营还没有要砍票的意思的样子,难道就这样放弃砍票了?如果是那样的话,那真是太好了。


\section{撕逼}
魔希回家陪绘里,邪教麻将终撕逼

      个人票力:200+ 厨团合计:500

大会战结束之后,数了数手里的票数。我们共拿了500左右,如果充分利用,加上些散票的话,居然超过了服务器众多的电磁……尽管事后知道有对手遭遇84的因素在,不过还是那句话,士气上得到了极大的鼓舞。同时,我们的核心团队也难得地得到了扩充——小夜加入了我们。

接下来的一场,基本上是麻将连爱丽丝为主,不过也没有太偏;对面仍然是炮邪联盟。但是后半场的时候,惊人的发现,有一些票被砍掉了!

运营——RUN桑的大砍刀就这样来了。不过这次被砍的不是麻将,而是电磁炮和邪教。晚上,就这样静悄悄地,绘里回家了。当时,真是百感交集,想着去年被砍回家的我和,再看看今年虽是对手但是同样被砍回家的绘里——尽管是对手。

不管怎样,比赛还是要继续的。由于电磁和邪教显著被砍,坊间开始流传电磁和邪教保持距离的流言。以及各种各样的原因吧,15日的比赛票面的比例,由肥希3:7爱妹,赛前临时调整为了1:9。我当时三次元事务比较多,也没在意外交什么的情况,甚至连票面都没看,只管拿票出票罢了。

直到晚上,突然发现情况不对啊。梨落和自由闹得好欢乐啊。于是翻查了一堆聊天记录,找到了自由和梨落的交易。哎哟哟,答应别人的事情,怎么说呢。但是事已至此,给梨落送票他也不要,也不知道该如何是好了。加上运营的砍票,肥希也只好回家陪绘里去了吧。

接下来的事情大家都知道了,本年度最劲爆的撕逼大戏。梨落和自由谁是谁非,我虽然觉得自由做得不好但也不好说太多;但是梨落和自由这个热闹的样子,不免被大家评论,甚至壕流说要出梨自的小说——果然萌战能收获的不仅仅有萌王呢!


\section{改变}
岭上开花本战毕,炮邪只剩一口气

      个人票力:250 厨团合计:500+

砍票,撕逼。萌战彻底被改变了。逐渐大家对砍票试探、适应,也进入了“新常态”。圆厨的小7还特意跑过来跟我们讲防止砍票的若干技术因素,可惜,我当时就知道她说的基本不正确。不过无论如何,还是要感谢小7的。

随着技术的完善,这边的票力也在逐步增强,挖掘机也由最初的单机改为双机挖掘,于是我们这边的剧本也改变了,甚至经常三位数地煮票。最初的24强占6还要“争取”,逐渐变成了占8、占10,甚至是——三桌麻将,半壁江山!

果然,接下来的邪教军团,一个个败下阵来,除了分组超好外加壕流关系的真姬以外。电磁炮呢,日子也不好过,至最后一场前,纷纷落马,唯有领袖炮姐,对手是他们“自己人”本子娜——顺利通过了。到了最后一场的时候,是宫永咲VS佐天泪子和九条可怜VS爱丽丝。

真的,我们都以为电磁已经放弃了。从开场一波来看,没有落在我们手里的票也没多少;更有之前的交换意向,这组保魔王通过的话,我们保泪爷复活——毕竟团体大战我们是强项。于是呢,只有常年通宵的枫叶在值守,挖掘机输出也设置地比较低,我们都去睡觉了。

第二天早上一看,顿时傻眼了。票面被虐了一大截,手里的票也没多少。被偷袭啦!这可怎么办?

除了加紧拿票以外,大概也没什么其他办法了。当时大家的想法是,“电磁总是要被砍的”,不过还是希望票面能拿下。这种场合,圆脸也支援不了什么,怕是对面电磁也是单枪匹马,没有邪教的支援了吧。和三回战第一场鸭子类似的场景出现了,电磁下午逐渐开始疲软,我们这边拿票却一切正常,于是,一边咬住票面,也逐渐攒够了足以逆转的海底。一切按着剧本来,晚上的时候,岭——上——开——花——!票面上直接逆转了电磁,当然砍票后就赢得更多了。

果然麻将和原著剧本一致呢。开场先锋战失利,最后大将扳回。中间出场的顺序,除了真子提前回家以外,中坚和副将的顺序也算正确。体现作品和角色特色,这也算是麻将的一大传统吧。

接下来的复活赛,接着捞马实在怕被砍。自由还去问了电磁,是否要捞泪爷,电磁的回答含含糊糊。管他呢,稍微帮忙捞一些吧,但是也没帮很多。这边圆麻合作仍然很融洽,拟定了若干打捞计划,倒是邪教居然没怎么出票,真是意外。

结果呢?泪爷票面2位直接被砍死,邪教更惨,只有自由带了些的日香好看些;眼看凛喵2票惨死,真是苦逼。圆麻计划打捞的各位差不多都捞上来了,甚至什么花音、结衣这样的——果然就这样成为圆麻萌了吗!


\section{挫折}
笨蛋真姬连被砍,壕流怒喊怎么玩

      个人票力:250 厨团合计:500+

季后赛分组,考虑到电麻和邪教基本要被砍死,圆麻协商了季后赛计划。经过若干轮分组讨论和合议,最终结论是——清场!

而且由于砍票的因素,到了这个阶段,最重要的已然不是票力的问题了。萌文,曲线形态,以及技术规避才是大头。于是,污染和追梦两位菊苣先后加入了我们,专门负责萌文相关事项。

另外,大概还是因为之前和邪教的撕逼吧,最终壕流离开了我们,这是最大的损失。同时,自由突然提出咲第一顺位,完全不顾之前一直被奉为精神领袖的乳和。我该怎么想呢?以前反复的征求内部意见确定弃保顺序,然后就这样被推翻了?呵呵。不管怎么说,还是先搁置争议共同发展,只能呵呵——果然自身强大之后,矛盾就来自内部了吗。

第一场整天也算顺利,基本上怜神和笨蛋都倍杀了爱丽丝。当大家都以为没问题了的时候,第二天早上却发现:笨蛋居然被砍死了!而受益者,又是第一次砍票的受益者——爱丽丝!这只是个巧合吗?尽管爱丽丝砍后“多出来”的几票确实是因为一开始的计票失误,但是这砍得也太偏了。

当时的第二场比赛进行中,我们带着乳和和龙华两个,一开始也算是压着打。可是傍晚真姬突然发力,又开始垂直,意想不到地逆一了!这回反正就这样吧,默认邪教是要被砍死的——所以结局一点也不意外,乳和一位晋级,龙华二位,真姬回家。

这真姬回家,估计壕流是预料到了的,毕竟之前有人作死,现在理所当然被运营针对;但是笨蛋惨死,绝对出乎几乎所有人意料之外——运营掌握着一切,壕流的两大本命相继惨死运营刀下!

还好,第三、第四局还算顺利,双麻顺利晋级。清场计划虽然遭到了破坏,但是还算不大离谱,但总是那么的忐忑不安。

壕流最后留下的话:“当年五千能赢的,现在看来几万也赢不了了。”


\section{荣誉}
砍票输赢荣誉在,仅剩圆麻战运营

      个人票力:250 厨团合计:500+

到了季后赛的后半段,第五场比赛便是作品名荣誉争夺战——当前仅剩的圆麻两大阵营,分别挂着作品名称的宫永咲和鹿目圆,终于有了一次直接的对决。虽然看着另两位“被捞上来的”东山就不难明白,这并不是一场生死大战,双双晋级是铁板钉钉的事情;可是毕竟关系到作品的荣誉,而且对麻将而言,也期待着第一场对圆的胜利。

真正比赛的时候,倒是还感觉良好,一直都维持着已发行半数以上的票,不发愁拿不下。甚至到了最后直接在已经领先的情况下玩儿大海啸,至少票面上完成了第一次对圆脸的胜利!

不过也别太高兴,要知道毕竟运营才是最大的。不管怎样,接下来便是AKO对杏子,以及炮姐的比赛了。会不会也像之前那场那样两家争一呢?不管怎样,二位也能晋级嘛。

结果还没开场,我们这边就掉链子了。北京时间23点,正是抓票的黄金时间,最重要的服务器却出现了问题。虽然奇迹般地找到了备机,但那也是半小时之后的事情了;同时部署这个那个忙了半天,总之就是一波颗粒无收。欲哭无泪啊。

开场的萎靡大家都看到了吧。还能怎样,后面多抓些。还是以往的老战术,和圆脸保持差距吧!猛然回头一看,却发现电磁也在后面不远处紧紧跟随。

逆了逆了!电磁把我们逆了。正当我们惊恐之时,没多久,电磁把圆脸也逆了!没办法,票面三位的我们只好玩儿命刷票,所以后面的曲线也不管了,票尾也不管了,压箱底的一大堆C票在53分的传统海底时间倾巢而出。最终票面,被打到3位的居然是圆脸!好吧,两家刷圆,无论如何,总算是二位满足了。

而到这个时候,运营连上一场的票都没砍完。这是放弃治疗的节奏?还是?不管怎样,AKO应该可以晋级了吧。倒是电磁,如果这一次不砍,那真是对坚持最好的回报了。


\section{等待}
运营砍票无余力,比赛继续不知期

      个人票力:? 厨团合计:?

赛后,圆厨小7来这边抱怨了一大堆掉链子——呵呵,原来掉链子也有伙伴啊。但是小7蛮悲观地认为运营不砍杏子回家,我却不这样认为。到了第七场的时候,由于还差两场没砍,我们考虑到为了保证圆脸最大概率相撞,这一场希望送麻美一位。

可是?!跑路啦,跑路啦!麻将的领袖自由,带着他的小姨子可怜,带着麻将的票跑路啦!真的这麻将完全没开场,麻将的开场票都刷给可怜啦!而且关于照姐和爱姐的弃保,也和本土产生了争议,本土甚至在票楼跳出来用中英日三语说“请投宫永照”!“Qingtoug Gongyongzhao”!还有拼音啊。于是这边紧急更换萌文,勉强还刷了个难看的曲线。

关于可怜的问题,第二天也是内部吵得可以。我们也看着自由海底时间两边都刷了不少,尽管我和枫叶几乎都投照姐,而且照姐也领先不少,不过最终结果还是要看运营了。这场,今年日萌第二号黄图厨和第一号黄图厨会不会合作,结果又会如何呢?

第八场我们如之前谈妥的不出手,于是圆脸自己刷得也好高兴的样子。瞧瞧人家,圆脸显著地区分了蓝黑的票力,真是为运营排忧解难啊。

等待的砍票结果,也只有第五场还好,第六场勉强砍完,炮姐自然是回家,但是这90%以上的多重率实在……哎,其实谁家不是呢。剩下的?呵呵!运营丢下一句“老子电脑坏了!”便遥遥无期。

眼看着这比赛休赛无事可做,这边新技术并没有停下来。为了利用越来越珍贵的萌文,这边在写萌文的同时,也改进了萌文的分配方法,尽量减少萌文的浪费。另一方面,票力方面,我这边的“挖掘机”继续前进,希望票力能再上一个台阶。

话说期间小7曾来找我咨询过刷票的问题,现在咨询这个作甚呢。我也想着可能是来套取情报的吧,嗯?不过好在对小7还算有些好感,透露了一些信息也没有说到关键,倒是我在小7的提醒之下考虑到了虚拟化技术,运用在“挖掘机”的话或许会有不错的效果吧。

可是随着时间流逝不安感越来越强。这服务器要续费,麻将今年的预算这看来是不够了呢。都是运营的错!更要命的是,19日,有人转发了大新闻,日本逮捕了向中国倒卖服务器的人,我们一登录发现发现服务器果然都不能用了!哎,真是前途未卜,等待,不安,何时是个头呢——明明还在胜利中,却想着邪教和电磁虽然已经回家,但总算是有个结局或许也不错了。


\section{拖延}
春日鸣锣盛夏战,秋风扫尽寒冬来

      个人票力:? 厨团合计:?

服务器完全不能使用,突然打破了之前的局面。说实话,要说大家公平地拼VG,我是一点都不虚的;不过缺少了服务器,海底放弃就算了,连开场的出票都有难度。

正在苦恼的时候,运营的新电脑好像终于到位了的样子,还是念念不忘砍票呢。如果有必要,出票大概还得像拿票一样的做法吧。实现出票策略需要时间,不过我倒是真心希望,如果运营能在大家都没有服务器的情况下继续比赛就好了——千万别我们没有服务器对面却有就好。

挖掘机的升级工作如期进行,保留一份旧版代码,新版上做了大规模的重构,效率显著提升,而且我们也不再需要2台挖掘机了——或者说调度2台挖掘机反而是个麻烦。新挖掘机,嘿,一台顶过去好多台!只是没有比赛,测试起来真是困难啊。

等到运营出手,照姐不出意料的死掉了,毕竟今年第一号黄图厨和第二号黄图厨的联袂演出;不过运营硬是把学姐的头(名)砍掉了,这确实让我感到非常意外——运营难道已经彻底崩坏,到了为了扶持新番还是自己的本命不要脸面的程度了吗?

于是又一次圆麻合作开始了,很可能是最后一次。双方联合约定,除非两个黄图一组,否则坚决不投黄图一票。想必如果黄图只有30票的话,怕是运营也不好太夸张吧。希望如此。

寒风瑟瑟,北京已经是寒冬了。想想比赛刚刚开始还是炎热的盛夏,这个赛程,到底什么时候是个头呢。

但是接下来几天,不断的延期教育了我们——居然相信运营,果然是太年轻了。这理由真是千奇百怪啊,从之前的电脑无法启动,到后来的网络故障,再有什么分组脚本无法启动……也无法分辨是真是假,定下的日程也一改再改,说是有史以来历年最差的运营也不为过了。我们呢,只能在各种服务器生死未卜、VPN又该续费等各种苦逼事之间挣扎,苦苦还是傻傻地盼着比赛继续吧。


\section{再开}
运营失策真爱隐,赛程混乱大厨烦

      个人票力:? 厨团合计:?

12月10日晚上,本来应该早就已经决出萌王的日子。但是这一天,最后一次分组结果刚刚发表。而且,还是使用的不知道什么黑科技的黑分组。

无论分组结果如何,比赛总算可以继续了,不要落个2014年没有萌王的苦逼结果。

不过分组结果,我个人还是相当满意的。一来两个黄图都遇到圆脸,而且如果挣扎出来也是四强之后的事情了;另外咲和可以到决赛才会师——当然这是我们的剧本。而且赛制也加入了大量的连记,在只有2家的情况下,谁人多呢?——21世纪,世界麻将人口已经数亿。

我另外的考虑是,灭黄图的时候圆脸会吸引大量的仇恨,特别是来自运营的仇恨。但是无论如何黄图也是会被剿灭的。接下来RUN桑,你是想让麻将再登顶呢,还是想让圆脸三冠+蝉联呢?虽然你可能是个圆豚。而且我和可以保送四强,尽管4进2会是一场硬仗,不管最终结果怎样,总胜场的记录应该可以保持很久了。

备受期待的圆鸭大作战,我想UMB自己的话大概会投鸭子吧。另外,由于运营的无能,第一轮要4连记。果然是千年修得CP连记啊,而且是在这么重要的赛场上。奇数组有龙怜,CP在场,即使是对黑也赢面大增;偶数组有咲和,果然是心灵相通一起摸红5P嘛(鸭子去死,AKO抱歉了)。接下来如果按我们的剧本的话,鸭子还可以和和来一次连记,顺便混个四强——也就这样,不过应该可以满意了。

再看看8个小组的顺序,除了TOKI“为了比较靠近龙华”而位置靠后外,其他简直是完美的“原著顺序”:先锋都在FT1组(TOKI找龙华去了没办法),次锋和副将在FT2组,大将在FT5-FT8组。那么希望结果也是——咲和会师决赛!于是我们的剧本:玄和红蓝怜鸭龙咲,就看运营让不让我们再次8占6了;接下来和红鸭咲或者和蓝鸭咲,最后咲和决赛后一起回家结婚——龙怜也一起回家,照淡也已然一起回家……

嗯嗯,可能我想得太美好了,不过不管怎么样,顺利重开就是好事,但是——真能这么顺利吗?


\section{崩坏}
伤厨破贴哪家恨,残楼败赛几多愁

      个人票力:350 厨团票力:450

确实如同贴吧上某段子一样,才一个月嘛,对麻厨来讲,休刊一次而已,非常感谢小林立老师!老师辛苦了!我们习惯了! 

日萌做到今天这个田地,毫无疑问是崩坏了。赛程表完全被践踏,身为运营单人黑幕审议不说以各种理由拖延比赛,中途休赛一个月,也是萌战史上第一奇葩了。

服务器仍然没有恢复。从我之前看到的几个报道来推断,看来短期内是恢复无望了。早就觉得那些拨号账号都来路不明,这下果然;既然都指明账号有问题了,那么找ISP联系账号主人修改密码也就是必要的步骤了——且算其他部分都是合法的话——于是要恢复最好的情况也要再次收集账号,这将是一个非常漫长的过程。

但是没有服务器对我们来讲反倒是一件好事,靠资金提升票力难度陡增。既然大家都没有服务器,那就在VG上拼命,挖掘机有绝对的优势。13日的龙怜连记对军火连记,麻将可以说相当弱的一个角色,面对对面实力最强、怨念最大的黑长直,简直是一场灾难;但好在隔壁有CP在,有爱就有奇迹!连记走起!

刚刚一鼓作气,开场就出了岔子。出票遇到问题,主板找不到票楼;搞了好长时间,才发现原来还是RUN的错——比赛休赛时间太长,票楼太旧了。于是处理了这个问题,一切继续。到早上又陆续解决了一些其他问题,尽管还有一些小问题,就如同大家看到的那样,顺利地进行下去了——直到结束,砍票前大腿赢了黑长直80票,同时作为另一半的怜神也报了多年的仇恨,至少砍票前战胜了学姐——萌王的对撞,扳回一局!

比赛结束,小7又跑到群里来发泄了一通。毕竟这个麻厨群是一个圆厨小7最初创建的,也算是比较亲密;我们的感慨是终于明白圆脸故事中蓝毛坏掉的原因了。

当晚两场比赛之间还见圆厨在圆脸的外群找人准备第二天的比赛,于是我们也开出了接近全力;第二天早上睡醒居然发现死群了!

唉。这个世界,就这样崩坏了吗。


\section{闹剧}
圆麻合战人疯狂,真假运营厨心慌

      个人票力:500 厨团票力:600

14日,16进8偶数半区。我方本篇男女主人公咲和、外战男女主人公鸭憧悉数登场。内战不表,毕竟和是精神领袖;外战鸭子对粉圆,令人期待的UMB大战。

早上的时候,我们鸭子的票面已经倍杀圆脸了。想来也是没什么压力了吧。

有uBJ什么的在运营楼高喊如果RUN到什么时候还不出来我们就这样继续比赛,我倒是非常同意这个办法,毕竟不能再拖下去了。2014没有萌王事小,恐怕要是真的停办,13岁的日萌就到此结束了吧。而且如果有人篡位,明年日萌运营也有了着落,RUN也能留点面子地退出——RUN你可知道圆麻早已商量好了不投黄图,让你怎么也砍不赢?

崩坏还在继续。运营楼渐渐闹起来了,先是有一个Ly什么什么在喊咲角色全部取消资格,然后麻将的本土意见领袖MAKO厨率领众和他对喷,更有更多好事者各种冒充运营给出不同的结果,还有不知何方神圣举报VG的IP——可惜作为日萌运营是没法封掉VG的,语言不就是对策吗,但也就这效果了——不过举报内容说参与者200人——呵呵,呵呵,呵呵。

简单的想法是Ly不过是一个疯掉的本土圆厨罢了;仔细想来,最受伤的应该还是RUN才对——这分明是各路群众讽刺RUN肆意篡改比赛结果嘛。

这么一整天下来,基本上维持着倍杀的节奏,一直战到终场——鸭圆票面622对311,整齐的2:1倍杀。下午的计划,今日个人500、鸭子600、鸭子倍杀,后两项完成,第一项差个位数票,就算500好了。

而且鸭子作为今年仅有的突破600的角色,而且是两破600,果然超神。上次鸭子出场运营缩了没砍,这次要是再不砍,那就彻底神鸭了——鸭子圆神化,圆神鸭子化,日萌一笑话。

这场结束,大家都当作日萌完结的感觉。挖掘机差不多就此公开了,小7也透露了一些圆厨的内幕;贴吧众却有好多人相信了那个闹剧结果,反观SOSG上根本没人理睬——今年日萌这场闹剧到了高潮,那也差不多快要结束了吧。

基于程序猿的本能我事后修理了一些场中遇到的问题,事实上还有1场外战嘛(如果日萌继续的话)。休刊不要紧,只要主义真!


\section{开黑}
运营开黑黑圆胜,麻厨挺咲咲开花

      个人票力:? 厨团票力:?

鸭子倍杀圆脸,结果300+的优势被砍到剩40,但好歹是一场胜利;再看另一组——可惜,龙怜CP双双进入8强的剧本,被带着大刀回来的RUN无情地砍掉了。197比195,黑圆胜,就“像”是刻意安排的一样。虽然说这也在意料之中吧!毕竟已经到了这个阶段,剩下的几个人不如大腿的可能也就只有小莳了;另一方面,还是之前所说,对面的对手却是吸散能力特强的黑长直——这样的比赛下来,就算有CP在场,输掉也是十分正常的事情。不过比较满意的事情是黄图双双回家,RUN大概是终于发现无论如何都没法把自己最爱的角色砍赢吧。

望着刚刚离场的憧不八,怀念明知会被砍还勇往直前的炮不八,再想想同样“黑”属性、已经多少年之前的汞不八——黑不八的称号被打破了。于是,剿灭黑长直这个艰巨而光荣的重任落,在了大魔王——“男”主角宫永咲头上,也落在了整个麻厨团队的头上。

这一次的失败使得麻厨再次团结起来。毕竟大魔王1挑3,想着好像鸭子也干过一次这种事——真是的,号称麻将的人数优势都哪儿去了?大厨light被请到了内群,一起负责萌文事宜。

联系到之前把“咲”加入NGWord的事情,不得不说RUN也够黑的了。现在黄图已然回家,运营自然只能在圆麻中挑选,但即使这样,我们仍然是吃亏的感觉。但是遇到黑幕又能怎样呢?去抗争,还是萎靡不前?再一次感到了之前电磁炮谢幕战的伟大,尽管他们输了,精神上非常值得尊敬。我们呢,自然也不能放弃,要把黑圆掐死在这一场,最好能把那个开黑的RUN从日萌开除出去!


\section{团队}
四大厨团厨组团,百家争萌争最萌

      个人票力:? 厨团票力:?

大家也都忙碌了起来,毕竟不仅仅要准备大魔王的票面,还有前一场的内战——无论如何,即使内战也要有个说得过去的票数,如果内战不管了的话,那不就是明着说——我们就是多重了嘛,请照着100票砍吧。尽管不重要的比赛可以少一些,萌文质量可以差一些,但是有一个基本稳定的票力输出,对维持形象来讲,也是非常重要的吧。

这个时候,我们的萌文队伍终于扩充到了3人。提高萌文质量,是应对运营砍票的重要手段吧。不过反过来说,既然运营是意识流砍还开黑,大概只要不是顽张*100,其他的也是RUN一个人说了算而已。这个时候,麻厨的团队除了萌文组,还有领袖自由、后期的土豪银狼,我这个挖掘机驾驶员以及负责“曲线修正”和夜勤的枫叶几人。虽然人数上没有显著增加,但是专业化的分工,使得效率显著提高。

同时,内战的话,还涉及到阵营内部协调的问题。想想多年前鸭子多次的惨死,以及还有外部未剿灭的二圆,维持一个稳定不分裂、有战斗力的团队是多么的重要。无论如何,今年对团队内部矛盾的控制是非常有效的,这也是一个非常好的因素。

看看今年海外几家的厨团,就我的理解来看,我曾在赛初考虑厨子和喜爱的角色的关系,写下这样的评价:
圆:鹿目圆-天地万物,神临天下
麻:原村和-理论计算,科学技术
邪:绚濑绘里-内政外交,连横合纵
炮:御坂美琴-哔哩哔哩,勇往直前

现在回过头来看,确实相当符合我当时的预期。圆脸还在战斗中,麻将是我自己的写照自然不会错;其他看来,梨落作为邪教的代表,基本上是靠各处外交、让票换场等手段的,尽管当时自由说LL的代表是日香日香日;至于以小9为代表的电磁炮,在最后一战不畏黑运营拼死一搏的表现,实在让我非常佩服,也非常符合美琴的性格特点呢。

仰望着炮厨的精神,我们也要继续努力!


\section{魔力}
四人麻将争四强,魔法少女战魔王

      个人票力:400+ 厨团票力:500

可是咲的比赛一开场,就遇到了岔子。由于没有妥善处理发行所被挤爆出现503的问题,直接导致丢了好多票。虽然凌晨补回来了很多并实现了反超,可是到了下午又遇到了VPN的不给力、拿票越来越不济——但是还有战友,还有团队,还有散票,再加上上午领先快200,最终票面还是170票胜出了——那准备的720票面还剩了好多……

比赛结束,仍然忐忑不安。怎么说呢,如果按照之前的比赛,咲有过小胜粉圆的记录,关键是那是RUN砍后承认的结果;而看上一轮,咲、黑被砍后都是197票——所以说还是一切尽在运营手中!再想想之前RUN近乎无耻地保黄图,这比赛到底该如何继续……

此时两场比赛待砍,运营楼uBJ的逼宫仍然在继续。“给RUN12小时啦”“强制继续吧”等等呼声不断,这也算是我们相当想实现的。可是RUN会怎么想呢,毕竟运营才是最大的。

1/4决赛前一场是4人麻将,结果运营直接来一句“这种比赛不知道该怎么砍”在运营楼,毫不意外地引起一片反弹。运营楼越闹越欢,关于砍票、日程,还有不知哪儿冒出来的“日语很奇怪”的家伙在投诉我们机械化刷票——可惜由于日语渣,那位也是自己碰了一鼻子灰而已。久久未出现的CODE屋似乎都被“炸”了出来,我们也开始有点担心挖掘机会不会受到影响,开始紧急准备人肉辅助验证码识别系统。

uBJ逼宫要求到时间强制开始,加上RUN自己不知为何放出那种声音,第一局迟迟才砍完——直到uBJ逼宫的第二局砍票截至时间已过,都到了下一场强制开始的时候,RUN终于姗姗来迟地出现了。


\section{封面}
咲鸭争和原著见,弱鸡麻将再逆天

      个人票力:? 厨团票力:?

“不砍了,就这么继续吧。”RUN淡淡地丢下一句话。

胜利!毕竟魔王,战胜了魔法少女!RUN怕是此时已然崩坏,不知如何继续吧。好吧,那我们就这样把这届比赛打完,毕竟只有2场了,还有一个敌人——蓝毛。

哪管他运营楼的热闹,更别提贴吧的喧嚣。认真准备下一场比赛,抓紧时间开场。这个局面,咲鸭争和,这应该就是麻将全国总决赛大将战的剧本吧,也是麻厨外群的封面;只是,找爸爸的小林立,不知何时才能画出——今天的日萌,却已经率先实现了。

刚开始拿票的时候,虽然第一波约了65票,但是日志里反复出现的Cookie冲突还是让我有些担心。尽管知道这可能仅仅是误报而已,不过还是觉得有些危险,做着大战的准备,在团队中召唤大家准备更多票面。

但是开场30分钟后,圆脸出乎意料的只有2票。投票版上一片对圆脸的嘲讽,看起来圆脸已经放弃治疗,我们可以提前庆祝胜利了吧!

想想当年一群新厨子、票力30票的弱鸡麻将,二回战兔子海爷隔空对战干翻了邪教,三回战神鸭600+打赢了电磁,FL阶段大魔王第一次取得了对圆胜利;再开后,借着没有服务器的东风,拿下了粉、黄,靠着圆麻协议淘汰了黄图;这次,终于连世界的主宰者——运营——这个“不可能战胜”的敌人都被打到放弃治疗了!如此励志而逆天的故事,不由得泪流满面——真的,有时候就是这样,生活中实实在在发生的事情,比那些少年漫画还精彩。

当晚的兴奋和喜悦,真的是难以表述。坚持的半年,奋斗过,等待过,煎熬过,喜悦过。一次巨大的成功,在日萌这个战场上,09年的没经验,10年的时运不济,12年的全面制霸;13年被运营审议致死,到现在14年不仅结束了对圆不胜的历史,更是获得了全面的胜利,实现了原著的结局。感慨麻将的历史,终于,写下了辉煌的新篇章,不过日萌,确实已经无法和当年想比了吧。


\section{会师}
四入四强荣耀在,一朝一位必可期

      个人票力:350+ 厨团票力:400+

明明已经是半决赛了,可这赛场却如同北京这个时间的天气一样寒冷——都是RUN的错,明明半决赛本应当是北京深秋的。随着圆厨的退场,可能连偶尔刮个大风的北京都不如了。

快结束的时候突发奇想想刷个海底玩儿玩儿,于是要了20张海底AA。毕竟没有服务器,我也只能做到这种程度了——想来如果专门为海底想想办法的话或许还能再提高,嘛,事已至此,没必要了嘛。

结果如同大家所见,398比48,这个胜利我相信即使砍票也无法动摇了。胜乳和者得萌王,这个多年以来传说的“萌王垫脚石”乳和,终于突破了自己3次4强不进决赛的悲剧,与自己的老公,最幸福地在决赛会师。想来战胜乳和的,09年的大河,10年的梓喵,12年的怜神,13年的YUNE,哪个不是最萌呢,呵呵。

尽管结束后仍然有人投诉我们机械化刷票和乱用萌文,不过呢,他撞到的IP是我RUSH用的,一般想来RUSH多重是比较困难的吧,嗯嗯。不管怎样,2ch上的舆论已经一边倒地支持麻将,看看讨论串已然是咲豚的狂欢,以及对圆脸超多重和运营不公正的讨伐。剩下的就是内战了,昵称咲日和;不过这只是攻受顺序,我倒是希望和能优胜——不过既然已经到了这个程度,其实咲和谁赢都好啦。

至于投诉中说到的用了小圆的萌文的事情,必须是之前准备的票面删掉了小圆。想来咲和这对夫妻决赛阶段连记了2次、外加最终会师,果然官配不可拆!


\section{结婚}
树上鸟儿成双对,夫妻双双把家还

      个人票力:? 厨团票力:?

若不是CODE屋撒手不管,若不是RUN挥起大刀砍死邪教和电磁,若不是日本警察在最恰当的时机查封所有服务器,若不是……大家的努力,以及各种机缘巧合,最终成就了今年麻将的霸业。生活就是这样,未知的每一天,比剧本更精彩。

咲和婚礼的当天,有不知何方神圣把发行所炸飞了。开场开始取票异常困难,想必这一天票数不会好看了吧。

凌晨刚开场的时候小和领先,整天曲线缠绵。晚上这位神圣把枪口转向了票楼,于是投票版要刷新多少次才能打开,尽管发行所已经恢复。在还有1小时的时候我刷了一堆和票,领先在15票左右。最后一波突然想起忘了出真爱,结果却拿错了票面——丢了一张咲的票出去。眼看着最后一波本土大厨疯狂刷咲的时候,甚至有人安慰我后年再来——

「嶺上ツモ。プラマイゼロです。」(“岭上自摸。正负零。”)

奇迹!奇迹出现了!187票平!不愧正负零大魔王,超人的强运,硬是把这“不可能”的奇迹实现了!夫妻共同分享这个萌王的称号,永世在一起,没有比这更棒的结局了!今晚,麻蜜尽情狂欢;今晚,麻厨彻夜无眠。RUN,你忍心砍掉这命运的羁绊吗?小林老师,能把12月23日定为咲和结婚的官设吗?

最后一局打完,心情过于激动,久久不能平静。六年的修行,终于换来了今天,这无比美满的双萌王,不仅战胜了运营,真是感天动地,怕真是剧本也不敢写如此结局吧。

今年一路走来,老公咲三回拿下了泪爷、季后赛刷爆了粉圆、半决赛把黑长直和运营打到放弃治疗,最终强行平票正负零——真的是功勋累累。老婆和战绩差些,基本上只赢了放弃抵抗的真姬和蓝毛,但仍然延续着24胜的超人记录。不管怎样,你们赶紧结婚!

礼单:挖掘机1台,军费4870元,代码万余行,萌文数千,大厨心血不计其数。

祝咲和百年好合,早得贵子。


\section{感谢}
半年艰辛胜百场,一朝萌王颂千秋

      个人票力:? 厨团票力:?

RUN最终还是知趣地没有出现。或许CODE屋和他聊了些什么。1小时发行,不限制CODE,好容易至少在外表上有起色的日萌,就这样了。

半年的萌战真的很漫长,半年也真的很短暂,但是有这么一段经历确实很精彩。纵使日萌衰退了,我们也曾努力过;纵使厨团解散了,我们也曾合作过;纵使以后再无来往,我们也曾共同奋斗过——萌史的长河中,有你,有我。

感谢所有真爱散,感谢你们撑起了日萌。

感谢小林立老师,感谢您创造了这些生动可爱的角色。

感谢自由,感谢你带领了麻将团队。

感谢小7,感谢你多次给我们帮助和支持。

感谢零件,感谢你的算法以及支持和鼓励。

感谢vx7@2ch,感谢你半年如一日在票楼支援麻将角色。

感谢まこ厨@2ch,感谢你为我们争取优良的本土舆论环境。

感谢壕流、Z壕(银狼)、小夜,感谢你们为麻将提供资金支持。

感谢追梦、污染、light,感谢你们为麻将提供萌文支持。

感谢红枫叶,感谢你为麻将刷票、修正曲线。

感谢津津、暖姐、勤桑,感谢你们的关心和鼓励。

感谢所有麻将外群的各位,感谢你们一直以来的支持。

感谢2ch的麻将讨论串,感谢你们默默提供萌文支援。

感谢所有支持麻将的人,感谢你们的爱。

感谢日本警察,感谢你们在恰当的时机查封了所有的服务器。

感谢uBJ@2ch,感谢你的逼宫推动比赛继续。

感谢CODE屋,感谢你为日萌做出的贡献。

感谢历年的麻将厨团,感谢你们种下这颗常青树,你们是值得尊敬的前辈。

感谢圆脸厨团,感谢你们对麻将一直的支持和培养,以及带来的精彩比赛,你们是值得尊重的朋友。

感谢电磁厨团,感谢你们勇往直前的精神,你们是值得尊重的对手。

感谢邪教厨团,感谢你们对我的激励。

感谢所有关注、支持、参加日萌的人,感谢你们。

(完)

\section*{后记}
\addcontentsline{toc}{section}{后记}

后记写得比较随便,和正文风格相差很大,可以认为并不属于总结的一部分。

本来,这是在RUN遥遥无期的休赛期间开始写的,毕竟太无聊。继续开赛以后,随着赛程的继续随比随写,完成了这24局故事;但后来情节越来越逆天(笑),渐渐地就当作小说来写了——与其说自由在写小说,我这个才是真正的当作小说在写。

可惜,毕竟是真实的事情、实在的人物,不得不说发挥起来有很大的局限性。尽管在当作小说的一开始就确定了必须基于事实的基调,但是即使是实话实说怕是也难免惹一些人不高兴。想来确实如此,去年我和被砍死,我们也确实经历过同样的感受。

而且,今年萌战可以说确实是技术碾压,对技术细节的处理,也是再三斟酌过。要说技术细节那我知道的可真的太多了,如别人所说,确实是专业,碾压别人好几圈也不是什么特别光荣的事情。但是能公开到什么程度,可真是一个很大的问题。考虑再三,修改再三,翻了翻之前别人写的总结——至少是在战吧出现过的内容吧,基本上以这一点为参考——出现过的我提,没有出现过的尽量模糊化。

还是尽量做人物的刻画和故事的描绘吧,这带着镣铐的舞蹈,甚至比高考作文更沉重的镣铐。能随便黑的人只有RUN而已(笑)。努力着,在12月22日,就这样修改、完善、小说化,预定在23日比赛结束后开始连载,并早早将前四局放到外群审稿,甚至还准备了2个结局——然后23日晚上把俩结局都删了,紧急新写了现在的结局,真的我自己也没料到这样的神棍剧情(笑)。

24日已经在外群公示了稿件,并随即送小7审阅,也是怕的让人不满意——毕竟,没发表之前修改,要方便得多。

中间出现了麻将的老前辈,我感到十分的意外和惊喜。毕竟,自己的成果得到别人肯定的时候,就是这样的高兴。全文也做了适当的小修改,但是以尽量不影响情节为目的吧。后来被小7投诉删文也是,哎,虽然也有私下交流,但是仍然感觉破坏了整体性很郁闷。

真实的事情,真实的故事就是这么精彩。但也因为他的真实,承载着虚构情节难以承载的重量。祝往后日萌还能再次兴旺,愿那些可爱的角色都能拿到理想的成绩。

但,一个人的登顶,就意味着一群人的遗憾。希望背后,总有着相应的绝望。

诸君努力。


