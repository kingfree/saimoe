\documentclass[UTF8, punct=kaiming, zihao=-4, a4paper]{ctexbook}


\usepackage{fontspec}
\setsansfont[BoldFont={PingFang SC Medium}]{PingFang SC Regular}
% \setmainfont[BoldFont={Songti SC Bold}]{Songti SC Light}
\setmainfont[BoldFont={SourceHanSansSC-Medium}]{SourceHanSerifSC-Light}
\setCJKsansfont[BoldFont={PingFang SC Medium}, ItalicFont={PingFang SC Light}]{PingFang SC Regular}
% \setCJKmainfont[BoldFont={Songti SC Bold}, ItalicFont={STFangsong}]{Songti SC Light}
\setCJKmainfont[BoldFont={SourceHanSansSC-Medium}, ItalicFont={STFangsong}]{SourceHanSerifSC-Light}
\setCJKmonofont[BoldFont={Kaiti SC Bold}]{Kaiti SC Regular}

\setCJKfamilyfont{libian}{Libian SC Regular} \newcommand{\libian}{\CJKfamily{libian}}

\newfontfamily\kaie{Kaiti SC Regular}
\setCJKfamilyfont{kaiti}{Kaiti SC Regular}
\newcommand{\kai}{\kaie\CJKfamily{kaiti}}

\newfontfamily\heie{SourceHanSansSC-Medium}
\setCJKfamilyfont{heiti}{SourceHanSansSC-Medium}
\newcommand{\hei}{\heie\CJKfamily{heiti}}

\newfontfamily\zhongse{SourceHanSerifSC-SemiBold}
\setCJKfamilyfont{zhongs}{SourceHanSerifSC-SemiBold}
\newcommand{\zhongs}{\zhongse\CJKfamily{zhongs}}

\newfontfamily\toppanbe{Toppan Bunkyu Gothic Demibold}
\setCJKfamilyfont{toppanb}{Toppan Bunkyu Gothic Demibold}
\newcommand{\toppanb}{\toppanbe\CJKfamily{toppanb}}

\newfontfamily\kashoe{YuKyokasho Yoko Medium}
\setCJKfamilyfont{kasho}{YuKyokasho Yoko Medium}
\newcommand{\kasho}{\kashoe\CJKfamily{kasho}}

\newfontfamily\minchoe[BoldFont={SourceHanSerif-Medium}]{SourceHanSerif-Light}
\setCJKfamilyfont{mincho}[BoldFont={SourceHanSerif-Medium}]{SourceHanSerif-Light}
\newcommand{\mincho}{\minchoe\CJKfamily{mincho}}

\newfontfamily\marue[BoldFont={DFMaruGothic-SB-WING-RKSJ-H}]{DFMaruGothic-Md-WING-RKSJ-H}
\setCJKfamilyfont{maru}[BoldFont={DFMaruGothic-SB-WING-RKSJ-H}]{DFMaruGothic-Md-WING-RKSJ-H}
\newcommand{\maru}{\marue\CJKfamily{maru}}

\newfontfamily\shingoe[BoldFont={ShinGoPro-Medium}]{ShinGoPro-Light}
\setCJKfamilyfont{shingo}[BoldFont={ShinGoPro-Medium}]{ShinGoPro-Light}
\newcommand{\shingo}{\shingoe\CJKfamily{shingo}}

\newfontfamily\outaie{OutaiKaiStd-Light}
\setCJKfamilyfont{outai}{OutaiKaiStd-Light}
\newcommand{\outai}{\outaie\CJKfamily{outai}}

\newfontfamily\tsukue[BoldFont={TsukuARdGothic-Bold}]{TsukuARdGothic-Regular}
\setCJKfamilyfont{tsuku}[BoldFont={TsukuARdGothic-Bold}]{TsukuARdGothic-Regular}
\newcommand{\tsuku}{\tsukue\CJKfamily{tsuku}}

\usepackage{multicol}
\pagestyle{empty}

\usepackage[usenames, table, svgnames]{xcolor}
% 黑白色
\definecolor{white}{HTML}{FFFFFF}
\definecolor{black}{HTML}{000000}
\definecolor{darkgray}{HTML}{333333}
\definecolor{gray}{HTML}{5D5D5D}
\definecolor{lightgray}{HTML}{999999}
\definecolor{midgray}{HTML}{666666}
% 基本色
\definecolor{green}{HTML}{C2E15F}
\definecolor{orange}{HTML}{FDA333}
\definecolor{purple}{HTML}{D3A4F9}
\definecolor{red}{HTML}{FB4485}
\definecolor{blue}{HTML}{6CE0F1}
% 日本色
\definecolor{sakura}{HTML}{FEDFE1}
\definecolor{ichigo}{HTML}{B5495B}
\definecolor{momo}{HTML}{F596AA}
\definecolor{kurenai}{HTML}{CB1B45}
\definecolor{butan}{HTML}{C1328E}
\definecolor{usubeni}{HTML}{E87A90}
\definecolor{nanoha}{HTML}{F7D94C}
\definecolor{enji}{HTML}{9F353A}
\definecolor{gusa}{HTML}{7DB9DE}
\definecolor{asagi}{HTML}{33A6B8}
\definecolor{mizu}{HTML}{81C7D4}
\definecolor{chigusa}{HTML}{3A8FB7} % 千草
\definecolor{ruri}{HTML}{005CAF} % 琉璃
% 作品色
\definecolor{saki}{HTML}{FEDFE1} % 咲 -Saki-, 桜
\definecolor{lovelive}{HTML}{91B493} % LoveLive, 薄青

\usepackage{geometry}
\geometry{left=2.4cm, top=2.6cm, headheight=.8cm, right=2.0cm, bottom=3.0cm, footskip=.8cm}

% \usepackage[toc]{multitoc}

\usepackage[math-style=TeX, vargreek-shape=unicode]{unicode-math}

\usepackage{pifont}

% \usepackage{fontawesome}

\usepackage{tabularx}

\usepackage{xhfill}
\newcommand{\xfill}[2][1ex]{{%
 \dimen0=#2\advance\dimen0 by #1
 \leaders\hrule height \dimen0 depth -#1\hfill%
}}
\newcommand{\xfilll}[2][1ex]{%
 \dimen0=#2\advance\dimen0 by #1%
 \leaders\hrule height \dimen0 depth -#1\hfill%
}

\usepackage{multirow}

\usepackage{ruby}

\usepackage{enumitem}
\linespread{1.4}
\setlist{partopsep=2pt, itemsep=1pt, parsep=0pt}
\setlist[itemize, 1]{align=left, labelindent*=1em, label={\makebox[1.8em][r]{\color{ichigo}\ding{166}}}, topsep={1pt plus 1pt minus 1pt}, itemsep={3pt plus 1pt minus 2pt}}
\setlist[itemize, 2]{leftmargin=0em, label={\color{gusa}\ding{101}}, topsep={2pt plus 2pt minus 1.5pt}, partopsep={3pt plus 2pt minus 3pt}, itemsep={1pt plus 1pt minus 1pt}, parsep={1pt plus 1pt minus 1pt}}
\setlist[enumerate, 1]{leftmargin=0em, labelindent*=.5em}
\setlist[enumerate, 2]{leftmargin=0em, labelindent*=.5em, topsep=0pt}

\usepackage{fancyhdr}
%\fancyhfoffset{0em}
%\renewcommand{\headrulewidth}{0pt}
%\fancyhf{}
\pagestyle{fancy}
%\fancyfoot{}
%\fancyfoot[R]{}

%\usepackage{tablestyles}
\usepackage{longtable}

\usepackage{colortbl}
\newcommand{\DataTable}[3]{{
\zihao{5}\renewcommand\baselinestretch{1.25}\selectfont
\ctexset{space=true}
\begin{longtable}{#2}% #2 = col
#3
% \centering
% \caption{#1}
% \label{#1}% #1 = caption
\end{longtable}
}}
\newcommand{\thead}{\bfseries \centering}
\newcommand{\codesu}[1]{\multicolumn{3}{l}{#1}}
\newcommand{\kokomade}[1]{\multicolumn{3}{l}{\xrfill{1.5pt}{#1}\xrfill{1.5pt}}}
\newcommand{\VoteFont}{\fontsize{11}{12}\kasho}
\newcommand{\VoteTabled}[1]{{\mincho\begin{tabular}{rrl}#1\end{tabular}}}
\newcommand{\VoteTable}[1]{{\begin{quote}\ctexset{space=true}\VoteFont #1\end{quote}}}
\newcommand{\VoteTables}[2]{{\ctexset{space=true}
\begin{longtable}{ll}
\begin{minipage}[t]{.37\textwidth}\kai 砍票前:\\\VoteFont #1\end{minipage} &
\begin{minipage}[t]{.59\textwidth}\kai 砍票后:\\\VoteFont #2\end{minipage}
\end{longtable}}
}\newcommand{\VoteTablet}[2]{{\ctexset{space=true}
砍票前:\\{\VoteFont #1}\par
砍票后:\\{\VoteFont #2}
}}
\newcommand{\Graph}[1]{\includegraphics[width=0.45\textwidth]{images/graph#1}}

\usepackage[hidelinks,unicode]{hyperref}
\hypersetup{
 pdftitle={Saki萌战应援},
 pdfauthor={王者自由},
 pdfsubject={2014},
 pdfkeywords={}
}

\makeatletter
\providecommand\phantomsection{}% for hyperref

\newcommand\listofillustrations{%
 \chapter*{引用}%
 \phantomsection
 \section*{插图}%
 \phantomsection
 \@starttoc{lof}%
 \bigskip
 \section*{表格}%
 \phantomsection
 \@starttoc{lot}}

\makeatother


\newcommand*{\plogo}{\fbox{$\mathcal{CP20}$}} % Generic publisher logo

\usepackage{graphicx} % Required for box manipulation

%----------------------------------------------------------------------------------------
%	TITLE PAGE
%----------------------------------------------------------------------------------------

\newcommand*{\rotrt}[1]{\rotatebox{90}{#1}} % Command to rotate right 90 degrees
\newcommand*{\rotlft}[1]{\rotatebox{-90}{#1}} % Command to rotate left 90 degrees

\newcommand*{\titleBC}{\begingroup % Create the command for including the title page in the document
\centering % Center all text

\def\CP{\textit{\Huge\zhongs Saki萌战应援}} % Title

\settowidth{\unitlength}{\CP\CP} % Set the width of the curly brackets to the width of the title
{\color{LightGoldenrod}\resizebox*{\unitlength}{\baselineskip}{\rotrt{$\}$}}} \\[\baselineskip] % Print top curly bracket
\textcolor{Sienna}{\CP} \\[\baselineskip] % Print title
{\color{RosyBrown}\Large\kasho\bfseries SAKI SAIMOE 2014} \\ % Tagline or further description
{\color{LightGoldenrod}\resizebox*{\unitlength}{\baselineskip}{\rotlft{$\}$}}} % Print bottom curly bracket

\vfill % Whitespace between the title and the author name

{\Large\kasho{王者自由\\[.5em](咲衣憧)}}\\ % Author name

\vfill % Whitespace between the author name and the publisher logo

\plogo\\[0.5\baselineskip] % Publisher logo
2017·5·1 % Year published

\endgroup}


\newcommand{\Madomagi}{劇場版~魔法少女まどか$\!\!$☆$\!\!$マギカ$\!\!$[$\!\!$新編$\!\!$]$\!\!$叛逆の物語}
\newcommand{\Mado}{魔法少女まどか$\!\!$☆$\!\!$マギカ}
\newcommand{\Saki}{咲-Saki-~全国編}
\newcommand{\Railgan}{とある科学の超電磁砲$\!$S}
\newcommand{\Kaminomi}{神のみぞ知るセカイ~女神篇}


\begin{document}

\zihao{5}
\tableofcontents

\ctexset{
chapter/name = {,},
chapter/number = \arabic{chapter},
section/number = {\arabic{chapter}.\arabic{section}},
}

\part[咲曲歌赏]{{\rm\sffamily\hei 天才麻将少女系列\\歌曲歌词鉴赏}}
\setcounter{chapter}{0}
\newcounter{kashicounter}
\renewcommand{\B}[1]{{\bfseries{#1}}}

\newcommand{\album}[1]{\def\albumName{#1}}%\addcontentsline{toc}{chapter}{\shingo #1}}
\newcommand{\songT}[1]{\stepcounter{kashicounter}\addcontentsline{toc}{section}{\mincho\arabic{kashicounter}. #1}}
\newcommand{\songTitle}[2]{\clearpage\songT{#1}
{\Large\maru\B{#1}\quad{\rm\small\outai #2}}\par}
\newcommand{\songTitled}[3]{\clearpage\songT{#1}
{\rm\tiny\mincho{#3}}\par
{\Large\maru\B{#1}\quad{\rm\small\outai #2}}\par}
\newcommand{\songMemo}[1]{{\rm\small\tsuku{#1}}\par}
\newcommand{\songText}[1]{\begin{multicols}{2}{#1}\end{multicols}}
\newcommand{\songTexts}[2]{\begin{multicols}{#1}{#2}\end{multicols}}
\newcommand{\newcolumn}{\vfill\null
\columnbreak}

\CTEXnoindent

\addcontentsline{toc}{chapter}{\shingo 咲-Saki-}
\album{Glossy:MMM}

\songTitled{Glossy:MMM}{橋本みゆき}
{咲-Saki- オープニングテーマ}
\songMemo{作詞:\B{畑~亜貴}/作曲・編曲:\B{虹音}
% TVアニメ「咲-Saki-」OPテーマ
}
\songText{\kasho
いまを抜けだそう\\
手に触れた Glossy future\\

熱くなれることが見つかった時は\\
迷わないで 止まらないで\\
素直になっちゃえ\\
荷物かかえ何処にしまえるかなんて\\
気にしないの 手ぶらで Hurry Hurry!\\

謎はパラレル 意味深パズル\\
負けないから 負けないよと笑う\\
楽しくなれたら勝ち\\

よしっ!\quad my mine Mind\\
広がる心の景色で会えた\\
君は自由に Fighting start!\\
悲しみを抜けだそう\\
触れた手を握りかえしたよ\\
道の先は\ruby{明日}{あした}に続くのかな\\
考えるより一緒に走ってみよう\\
\newcolumn
冷めた気持ちなんて自慢にならない\\
かっこわるいね ちょっとずるいね\\
本音があるでしょ\\
みせてごらん誰か信じてみたいと\\
隠したって 駄目だよ Bunny Bunny!\\

雨にパラソル 駆け込むパレス\\
休めるとき 休まなくちゃ次は\\
壁を乗り越えるんだ\\

もしっ?\quad my mark Mad\\
常識外れた答えになっても\\
私の前で Don't stop it!\\
嬉しいこと生みだそう\\
「間違えた」それがどうしたの\\
予測外れ逆転に変わるのかも\\
感じるままに裸足で大胆に行こう\\

よしっ!\quad my mine Mind\\
広がる心の景色で会えた\\
君は自由に Fighting start!\\
悲しみを抜けだそう\\
触れた手を握りかえしたよ\\
道の先は\ruby{明日}{あした}に続くのかな\\
考えるより一緒に走ってみよう\\
}

\album{熱烈歓迎わんだーらんど}

\songTitled{熱烈歓迎わんだーらんど}{清澄高校麻雀部\\
% \eighthnote
宮永咲(植田佳奈)・原村和(小清水亜美)・片岡優希(釘宮理恵)・染谷まこ(白石涼子)・竹井久(伊藤静)}
{咲-Saki- エンディングテーマ}
\songMemo{作詞:\B{畑~亜貴}/作曲:\B{福本公四郎}/編曲:\B{安藤高弘}
% /TVアニメ「咲-Saki-」EDテーマ
}
\songTexts{3}{\fontsize{10.5}{10.5}\kasho
がんばっちゃった がんばった我々\\
東南西北 わーいわーい\\
集まれ! わんだーげーむ始めましょう\\

(はいっ牌 つー牌 はいっ牌 一向聴!\\
~はいっ牌 つー牌 はいっ 多面聴!)\\

本当に悔しければ\\
泣くだけなんてムダです\\
こころは次の場所へ走り出したよ\\
(挽回逆転)\\

来来・未来は(来来対面)\\
来来・真っ白(来来混一色)\\
自分次第ね(ダブ東ドラドラ流して)\\
ちゃっかり上がろ\\

やる気だキミ 強気でキミ\\
楽しくなるね\\
勝ち負けよりも(もーいっかい)\\
駆け引きさせて(もーいっかい)\\
その気のキミ 陽気なキミ\\
明日はもっと\\
一万点 十万点 百万点の世界へ\\

がんばっちゃった がんばった我々\\
聴牌即立直 わーいわーい\\
集まって! わんだーげーむ続けましょう\\

(ポン~チー~カン~ロン はいっ牌 双ポン聴!\\
~ポン~チー~カン~ロン はいっ 中張牌!)\\

優しくなぐさめても\\
成長なんかしません\\
転んだままじゃなくて立ち上がるのよ\\
(根性全開)\\

謝謝・感謝の(謝謝一荘)\\
謝謝・言葉は(謝謝翻牌)\\
早めに投げよう(リーズモドラドラそのあと)\\
風向き変わる\\

見つけたユメ 探したユメ\\
遊びのなかで\\
指が踊れば(さーおいで) 熱烈歓迎(さーおいで)\\
感じてユメ 念じたユメ\\
かたちがあるよ\\
長方形 正方形 三角形のどれかだ\\

はりきっちゃった はりきった諸々\\
聴牌即立直 ほーいほーい\\
あるんだって! わんだーらんど向かいましょう\\

(はいっ牌 つー牌 はいっ牌 一向聴!\\
~はいっ牌 つー牌 はいっ牌 多面聴!)\\

来来・未来は(来来対面)\\
来来・真っ白(来来混一色)\\
自分次第ね(フリテンリーチで流して)\\
ちゃっかり上がろ\\

やる気だキミ 強気でキミ\\
楽しくなるね\\
勝ち負けよりも(もーいっかい)\\
駆け引きさせて(もーいっかい)\\
その気のキミ 陽気なキミ\\
明日はもっと\\
一万点 十万点 百万点の世界へ\\

がんばっちゃった がんばった我々\\
東南西北 わーいわーい\\
集まれ! わんだーげーむ続けましょ\\
(ハイハイハイハイハイハイ)\\
終わらない\\
(ハイハイハイハイハイハイ)\\
終われない\\
ジャンジャンジャランジャン\\

(ポン~チー~カン~ロン~わいわいっ~もーいっかい!\\
~ポン~チー~カン~ロン~わいわいっ~さーおいで!)\\

(熱~烈~歓~迎~ほいほいっ~わんだーげーむ!\\
~熱~烈~歓~迎~ほいほいっ~わんだーらんど!)\\
}

\songTitled{残酷な願いの中で}{宮永咲(植田佳奈)・原村和(小清水亜美)}
{咲-Saki- エンディングテーマ}
\songMemo{作詞:\B{畑~亜貴}/作曲:\B{rino}/編曲:\B{虹音}
% /TVアニメ「咲-Saki-」EDテーマ
}
\songText{\kasho
あきらめたら終わり\\
気持ちをリセットして 戻る場所で\\
One more play\\

今ちょっとだけ 目を閉じたら\\
せめて私が落ち込んでるの隠せるかな\\
ゴメンあんなに 応援してくれたのに\\
つまずいた 自分が許せないの\\
階段をのぼるたびに\\
誰か置き去りにされるとしたら\\
悲しみ受けとめるよ\\
ここでやめられないと立ち上がる\\

ただひとつ願い抱えて\\
それぞれの運命 賭ける負けないよと\\
ふるえながら\\
でもあきらめたら終わり\\
気持ちをリセットして 巡り合いは\\
残酷すぎてこわい…\\
\newcolumn
上向いてみよう 強がりでも\\
いつか自然に微笑んでると感じるから\\
オーライこんどは 反省ムダにはしない\\
落とし穴 楽々飛び越えちゃって\\
悔しさにそまるこころ\\
全て生かせればチャンスに変わる\\
もいちど考えるよ\\
ここでくじけるなんてらしくない\\

また進む時間の先で\\
君たちに会うでしょう 次は勝ちたいから\\
きずついても\\
いつ始まるかは知らない\\
高度なトリックより 空っぽにした\\
意識がついに動く…\\

あきらめたら終わり\\
残酷なルールで 叶う望み(立ちあがれば)\\

ただひとつ願い抱えて\\
それぞれの運命 賭ける負けないよと\\
ふるえながら\\
でもあきらめたら終わり\\
気持ちをリセットして 巡り合いは\\
残酷すぐてこわい… \\
}

\album{bloooomin'}

\songTitled{bloooomin'}{Little Non}
{咲-Saki- オープニングテーマ}
\songMemo{作詞・作曲:\B{Little Non}/編曲:\B{安藤高弘\&~ Little Non}
}
{\fontsize{10}{10}\kasho
\begin{multicols}{2}{
  上げってんの Wow! Wow!\\
  上がってんぞ Yeah! Yeah!\\
  上げってんの Wow! Wow! Yeah!
  \newcolumn
  上げってんの Wow! Wow!\\
  上がってんぞ Yeah! Yeah!\\
  上げってんの Wow! Wow! Yeah!
  }\end{multicols}
  \vspace{-1.4em}
哀話だ柔和に戦闘 包容だ そういうの どうよ 正す?\\
愛敬で疲労キープ 妥大人 OSミス古い$\!$※$\!$注エラー\\[-.3em]

顔にラミネートしても 感情!すっぴんまた養老で愚痴\\
解放するタイミング 曖昧は 遠に 捨てるさ 今走るさ\\[-.3em]

On the road!\\
生粋 永久にずっとロック 試練も どんと来いって歓迎するよ\\
自愛美 行こう! 決めた! In my dream\\
古辰 今すぐ速攻ダッシュ! 努力も 腐敗しない強情も全部 捨て身 改革心\\[-.3em]

きっと いつか見える果てまで 笑顔の花 咲かせたい\\[-.3em]

上げってんの Wow! Wow!\\
上がってんぞ Yeah! Yeah!\\
上げってんの Wow! Wow! Yeah!\\

やっと忠告、殷鑑不遠だ 原因は常 愛護は迂遠だ生一本!気取る\\
却下され乱心自我廃棄 あぁ…無情 夕冷えに一人聞くロックンロール\\[-.3em]

結局、湾曲に秀でても 哀調な周期不意に不安で放浪\\
尸位素餐 楽に素勤務 そんなんじゃ駄目出しわかってる目をそらすな\\[-.3em]

bloooomin' now!\\
忍耐 暗い響きだってGood! 涙は もっと出るさ限界じゃない\\
素直さ愛そう 自彊やめず Keep on dream\\
地音 友情も邪推してちゃ 見えない 本意はるか遠く善戦無縁だ 急げ 改革心\\[-.3em]

追いかける夢の先 信じる事止めないよ\\[1em]

激声で好きって言うのは延長だ 感に堪えないBEST TAKE 絶対にするまで\\
不穏に引くテンション 負けない 明快 強引 苦労人で驀進さ\\[-.3em]

On the road!\\
生粋 永久にずっとロック 試練も どんと来いって歓迎するよ\\
自愛美 行こう! 決めた! In my dream\\
古辰 今すぐ速攻ダッシュ! 努力も 腐敗しない強情も全部 捨て身 改革心\\[-.3em]

きっといつか見える果てまで 笑顔の花 咲かせたい\\
無限大 未知の果て 作り出す道を行く

% 上げってんの Wow! Wow!\\
% 上がってんぞ Yeah! Yeah!\\
% 上げってんの Wow! Wow! Yeah!\\
}

\album{四角い宇宙で待ってるよ}

\songTitled{四角い宇宙で待ってるよ}{清澄高校麻雀部\\
% \eighthnote
宮永咲(植田佳奈)・原村和(小清水亜美)・片岡優希(釘宮理恵)・染谷まこ(白石涼子)・竹井久(伊藤静)}
{咲-Saki- エンディングテーマ}
\songMemo{作詞:\B{畑~亜貴}/作曲・編曲:\B{木下智哉}
% /TVアニメ「咲-Saki-」EDテーマ
}
\songTexts{3}{\fontsize{10}{11}\kasho
わたしもしたいよ\\
ココロは正直 ドキツモ快感\\
手加減しないで\\
四角い宇宙で テンパイすちゃらか\\
テンパイ即リー快感すちゃらか\\
青春当然ハコテンぶっとび\\
クイタンあとづけ連チャンサクサク\\
ハイパイいーしゃん天国らりほー\\

当たって(誰でもこいな) 狙って(何でもこいな)\\
和了りはどんでん返し(リーヅモウラウラごめーん)\\
いけないわ ホンキだして(半チャンもういっちょ!)\\

読みって(オカルトないし) 打ちって(味方いないし)\\
流れはもってかなくちゃ (追っかけリーチいゃーん)\\

足りないわトキメキ\\
疼くような面子がどうしたの?(トイトイトイトイ)\\

わたしとしょうよ\\
ココロが晴天 ドキツモ昇天\\
手加減しないで\\
四角い宇宙で テンパイらりるれ\\

テンパイ即リー快感すちゃらか\\
しーさんしーさんしーさんぷたぷた\\
焦燥先ヅモかんちゃんずっぱし\\
満場感動焼き鳥泣き泣き\\

終わって(泣いてもろんりー)\\
変わって(笑えばろんりー)\\
強運信じてみるかい(タンピンサンシキすごーい!)\\
これからよ キモチこめて(連チャンもういっちょ!)\\

駄目って(あきらめないし) 迷って(投げださないし)\\
粘りの神髄見せちゃう (何でもかんでもとぉーし)\\

指先にヒラメキ\\
才能プラス努力で勝負でしょう?(タンタンタンタン)\\

あなたとしたいよ\\
チャクラが全開 ワルマチ満開\\
ぶつけるつもりよ\\
わたしの宇宙は ションパイはれほれ\\

テンパイ即リー快感すちゃらか\\
純情アリアリワレメでぶっとび\\
ひんしゅくリーのみウラドラ3発\\
愛情親リー高いよおりてね\\

スカートひらりと(ダメンチャンなら)\\
ドキドキドキツモ(ウキウキです)\\
髪の毛さらりと(ドラアマコで)\\
ズキズキツモツモ(しあわせです)\\

疼くような面子がどうしたの?(チーチーチーチー)\\

わたしもしたいよ\\
ココロは正直 ドキツモ快感\\
手加減しないで\\
四角い宇宙で テンパイすちゃらか\\

テンパイ即リー快感すちゃらか\\
青春当然ハコテンぶっとび\\
クイタンあとづけ連チャンサクサク\\
ハイパイいーしゃん天国らりほー\\

みじかいスカートも(カラテンなら)\\
なーがいスカートも(かなしいです)\\
みじかい髪でも(現物なら)\\
なーがい髪でも(安心です)\\

スカートひらりと(ダメンチャンなら)\\
ドキドキドキツモ(ウキウキです)\\
髪の毛さらりと(ドラアマコで)\\
ズキズキツモツモ(しあわせです)
}

\addcontentsline{toc}{chapter}{\shingo 咲-Saki- THE~夢のヒットスクエア~キャラソン対局編}
\album{咲-Saki- THE~夢のヒットスクエア~キャラソン対局編}

\songTitled{麻雀天使にかこまれちゃう}{\\
宮永咲(植田佳奈)、天江衣(福原香織)、池田華菜(森永理科)、加治木ゆみ(小林ゆう)}
{\albumName}
\songMemo{作詞:\B{畑~亜貴}/作曲・編曲:\B{木下智哉}}
\songText{\kasho
わかる わかる\\
とっておきのREVOLUTION\\
キミもどうだい?\\

なんと今日の挑戦は\\
初めましてなんかいらない\\
全部いただきまーす\\

遠慮だってついに消えちゃうよ\\
だからもっと運の真髄と\\
そうだ 戯れよう?\\

この世に嵐 ここだけ雷雨\\
ワクワクしてきた 大興奮!\\

時間は当然流れるけど\\
ねー?\\
関係なくなるから\\
快感体験まだでしょ\\
キミは キミは やってみたいはず\\

天和なんてやってみたいし\\
緑一色は美しいし\\
九蓮したって 死んじゃわないよ\\
配牌だけじゃない\\

毎回違うセカイ\\
楽しもうよ ネギしょって来い\\

よいね天の混乱が\\
招きますよちょいと地獄\\
決心ついたかなー\\

平和だったそうね大体ね\\
ここにあった牌の存在は\\
恵みか 罠か?\\

握る幸せ 捨てて絶望\\
ゾクゾクするよね 熱中戦!\\

終わりは絶対ゆずれないよ\\
もー!\\
同情する余地なし\\
なーい!\\
完全完璧だからね\\
つぎは つぎは こっちに立つはず\\

地和なんてやったことあるし\\
大三元は よくあるテンパイ\\
清老だって できそうだよ\\
自摸牌だけじゃない\\

対局それがミライ\\
面白いよ ダシとって来い\\

時間は当然流れるけど\\
ねー?\\
関係なくなるから\\
おーい!\\
快感体験まだでしょ\\
キミは キミは やってみたいはず\\

人和だってやったことあるし\\
小四喜は よくあるテンパイ\\
字一色も合わせたいよ\\
配牌だけじゃない\\

毎回違うセカイ\\
楽しもうよ ネギしょって来い
}

\songTitled{予感、咲きました!}{宮永咲(植田佳奈)}
{\albumName}
\songMemo{作詞:\B{畑~亜貴}/作曲・編曲:\B{村井~大}}
\songText{\kasho
探せないな迷子の\\
素直な気持ちはどこ?\\
あの頃から逃げたのかもしれない\\

しかたないね小さな\\
心のなかでひとり\\
遊ばないって決めてたよ\\

それでも輝いて\\
進むみんながいた\\
私だって 私だって\\
夢に会いたい\\

想いが開くとき\\
熱い奇跡の花\\
四つの花びら舞い降りて来るよ\\
指先に予感で\\
踊る 今の今が\\
私を待ってたの? この場所が謎の欠片\\

見つけたいと本気で\\
祈れば悲しみから\\
浮かび上がる怖くなって泣きそう\\

むずかしいね過去から\\
思い出目覚めなさい\\
勇気もっと咲かせるよ\\

臆病な眼差しじゃ\\
先に行けないから\\
強くなって 強くなって\\
夢を飛びたい\\

いつでも偶然と\\
言われるだけじゃなく\\
四つが一つの光呼ぶ景色\\
描いてみせるのが\\
不思議 誰と誰を\\
私は待ってるの? この場所で決めてしまおう!\\

想いが開くとき\\
熱い奇跡の花\\
四つの花びら舞い降りて来るよ\\
指先に予感で\\
踊る 今の今が\\
私を待ってたの? この場所にいると決めた!
}

\songTitled{刹那の海よ}{天江衣(福原香織)}
{\albumName}
\songMemo{作詞:\B{畑~亜貴}/作曲・編曲:\B{村井~大}}
\songText{\kasho
海底 海底\\
此の場 崩壊 崩壊 闇の真価\\
流局 流局\\
何れ月光参じて 刹那歓喜\\
西入 西入\\
誰ぜ神の路 覗くまいて\\
降臨 降臨 俗の世へと\\

9600 闇聴 怯えの手\\
槓裏 マルノリ 救い降る\\
海底撈月 見て居れよ\\
さぞ さぞ 苦しかろう\\

幽谷に\\
ひとりぼっち 退屈があるのみ\\
ひとりぼっち 戯れに興じて\\
身を 身を 滅ぼすまで\\
悟り得ぬ定めに\\
ひとりぼっち 仮初めの友でも\\
ひとりぼっち 愚かしい笑みでも\\
気が 気が 浮かれるまま\\
人に交わりて 暫し留まれり\\

即リー 即リー\\
彼の地 生還 生還 希望皆無\\
連荘 連荘\\
誉れ 同時に奉ずる 刹那頂点\\
二翻縛り 八連荘\\
誰ぞ鬼と化し とどめ刺して\\
茫然 茫然 暗き永遠よ\\

現物 ベタオリ 最果ての\\
八連しそうな 光など\\
オーラスリーチも 届かぬよ\\
あな あな 口惜しかろう\\

単簡な\\
あそびあいて 漁火焚きあげろ\\
あそびあいて 呼ぶ業の易きよ\\
目と 目が 惹かれ良かれ\\
言の葉でいざなう\\
あそびあいて 待ち続けた日々は\\
あそびあいて おののく者ばかり\\
其は 其は 見つけたのか\\
人は面白き 波を起こすなり\\

18000 直撃 震える手\\
表裏 マルノリ もう一度\\
海底撈月 汝等には\\
さて さて 会うべきだろう\\

有終の\\
ひとりぼっち 退屈がまさかの\\
ひとりぼっち 戯れに変じて\\
身を 身を 預けるなら\\
祭りへと急ぎて\\
あそびあいて 待ち続けた日々よ\\
あそびあいて 別れを告げるべき\\
気が 気が 浮かれるまま\\
人に交わりて 人は友を得る
}

\songTitled{イキナリナリユキナリッ}{池田華菜(森永理科)}
{\albumName}
\songMemo{作詞:\B{畑~亜貴}/作曲・編曲:\B{tetsu-yaeh}}
\songText{\kasho
「楽しんで」って言われたよ\\
そのコトバと 歩いてゆこッ\\
前向いて 上を向いて\\
ナリユキ次第やるんだ\\

あれ? あれれ? 字牌ばかり\\
あれ? あれれ? 切るとく〜る\\

まわりが魔物なら\\
あたしは退魔師かいッ\\
キリキリ舞々 ありゃなんなんだー!\\
ちょーし出ない\\

まわりの空気まで\\
あたしを迷わせた\\
ムリムリ精々 ハッタリかまして\\
しがみつくぞ\\

まーたヒドイ あぶないよあぶない\\
自分らしさ? あーッ テンション落ちるよ\\

「楽しんで」「へコまないで」\\
そのコトバを もいちど言って\\
前向いた 上を向いた\\
イキナリきたよキツイッ\\
「和了れない」「さ、どーするのさ?」\\
アタマがゲンカイ アイツらサイテー\\
テンパるも 役がなくて\\
イキナリ今日はつらいな\\

あれ? あれれ? ちーちゃばかり\\
あれ? あれれ? じごなのね\\
あれ? あれれ? 字牌ばかり\\
あれ? あれれ? 切るとく〜る\\

くやしい動けない\\
だれかが魔術師かいッ\\
ジワジワ続々 こりゃおかしいよー!\\
チカラ抜ける\\

くやしく終わるのかッ\\
だれかに呪われた?\\
ヤミヤミ津々 サッパリわかんねぇ〜\\
みんなごめん\\

うーわイカン まけそうだまけそう\\
降参しない! ねーッ チャンスつかんで\\

「楽しめない」「悪い夢だ」\\
ふるえる足 あたしの抵抗\\
ナミダには 早いよほら\\
ナリユキ次第やるって!\\
「邪魔すんな」「じゃ、どーなるのさ?」\\
ココロがホーカイ くるしージョウキョウ\\
あたり牌 しょせん数枚\\
ナリユキ次第だめかな\\

「楽しんで」「へコまないで」\\
そのコトバを もいちど言って\\
前向いた 上を向いた\\
イキナリきたよキツイッ\\
「和了れない」「さ、どーするのさ?」\\
アタマがゲンカイ アイツらサイテー\\
結局は ひとりノーテン\\
ヤッパリ今日はつらいな\\

あれ? あれれ? ちーちゃばかり\\
あれ? あれれ? じごなのね\\
あれ? あれれ? 字牌ばかり\\
あれ? あれれ? 切るとく〜る
}

\songTitled{見えない君の探し方}{加治木ゆみ(小林ゆう)}
{\albumName}
\songMemo{作詞:\B{畑~亜貴}/作曲:\B{田代智一}/編曲:\B{安岡洋一郎}}
\songText{\kasho
ここまで やっとここまで\\
来たもちろん 簡単じゃない\\
あの頃 探せど探せど見えない君\\

戦いを共にしてみたいオーラ\\
受け取るこころが騒ぎ出す\\
扉を開ける 答えを持たず\\
思わず叫んだよ\\

聴牌なら君も\\
ゆらり出る現る\\

誰でも きっと誰でも\\
武器はまだ 隠しておくさ\\
鍛えろ そっと鍛えろ\\
武器になれ 秘密のちから\\
ここまで やっとここまで\\
来たもちろん 簡単じゃない\\
ヒキとか ツキとか越えたい\\
こころを動かす駆け引き…\\
\newcolumn
仲間さえ忘れ特別な能力\\
磨いてますます溶け込んで\\
孤独は味方 自分を守れ\\
必ず勝てるから\\

配牌から希望\\
さらり受け赴く\\

探せど ずっと探せど\\
気配つかむ 手がかりは無し\\
あの頃 知ったあの頃\\
決めていた ちからが欲しい\\
ここまでやっとここまで\\
来たもちろん 偶然じゃない\\
鳴きとか 槓とかしないで\\
こころを覗けば魔物が…\\

誰でも きっと誰でも\\
武器がまだ 眠っているさ\\
鍛える そっと鍛える\\
武器になれ 秘密のちから\\
私は もっと君への\\
驚きを 伝えるために\\
あの頃 探せど探せど\\
見えない 君へと 叫んだ
}

\songTitled{Angel zone}{原村和(小清水亜美)}
{\albumName}
\songMemo{作詞:\B{畑~亜貴}/作曲:\B{福本公四郎}/編曲:\B{安藤高弘}}
\songText{\kasho
夢と現の境目を\\
くぐり抜ける 私が見えた\\

冴えて目覚めて火照り出す\\
軽くなる身体は\\

Angel zone 越えてる\\
もう誰もいないの\\
ふわっと ふわっと 大きいね\\
光った雲に Touch\&kiss\\

決めましょう 迷彩で\\
行きましょう ひっかけ\\
あなたといつも一緒にいたい\\
決まりです! ダマテン\\
行くべきです 直\\
絆の糸(切れない糸)感じてますよね?\\
My soul friend\\
\newcolumn
地上も天もそこにある\\
望むたびに 浮かび上がるの\\

怒る悲しむ抱きしめる\\
隠せない気持ちで\\

Angel style 遠くで\\
あの輝きは何?\\
わかった わかった やっぱりね\\
飛ばした合図 Touch\&catch\\

好きでしょう 危険牌\\
燃えましょう 三家和\\
あなたはもっとしっかり咲くわ\\
好きこそっ! おっかけ\\
燃え足りない オーラス\\
明日はまだ(これからなの)信じてますから!\\
I hope all\\

行きましょう 親リーで\\
決めましょう リーソク\\
あなたといつも一緒にいたい\\
決まりです! 壁だから\\
行くべきです とおし\\
絆の糸(切れない糸)感じてますよね?\\
My soul friend
}

\songTitled{逃しません…ですわ!}{龍門渕透華(茅原実里)}
{\albumName}
\songMemo{作詞:\B{畑~亜貴}/作曲:\B{田代智一}/編曲:\B{近藤昭雄}}
\songText{\kasho
目覚めなさい 目覚めなさい\\
はじめなさい はじめなさい…ですわ!\\

そうね わたくしになってみたい?\\
勘違い まぁいいわよ アイドルの悩み\\

常にヒロイン 勝利しか\\
味わいたくありませんの\\

強い方 求めますわじっくり\\
あーら 貴女たちいかが?\\
冷静な指で 確かめたくなる\\
興奮と刺激のステージ\\

思惑が外れたらピクピクと\\
つい短気な癖で\\
入れば リーチですーわ\\
おふざけもたいがいに\\
あ〜〜 しくされですわーッ!\\
デジタルモード 無視\\
わたくしは賭ける まだまだ\\

目覚めなさい 目覚めなさい\\
挑みなさい 挑みなさい…ですわ!\\

なあに わたくしはデンジャラス?\\
噂では さぁどうなの 目立つのね美貌が\\

立てばミネルヴァ 勝負では\\
華麗に打つ乙女ですの\\

怖い役 素敵ですわうっとり\\
ねーえ お楽しみいかが?\\
欲望に駆られ 余裕のつもりが\\
オーラスで主役はトラブル\\

プライドを潰せるわゾクゾクと\\
ほら夢の終わりが\\
親倍 8000オール\\
だれよりもなによりも\\
あ〜〜 目立ちますわーッ!\\
デジタル対決 無我\\
わたくしのために うてうて\\

強烈な爽快感でワクワクと\\
盛り上がる想いは\\
直撃 狙い撃ちですわ\\
おふざけもたいがいに\\
あ〜〜 しくされですわーッ!\\
デジタルモード 無視\\
わたくしは賭ける まだまだ\\

逃がしません 逃がしません\\
はなしません はなしません…ですわ!\\

お首洗いお待ちなさーい!!
}

\songTitled{ひとりにひとつ}{福路美穂子(堀江由衣)}
{\albumName}
\songMemo{作詞:\B{畑~亜貴}/作曲・編曲:\B{前田克樹}}
\songText{\kasho
つらかった事など忘れましょう\\
次からは気分を変えて がんばれる\\

だってね ひとりにひとつ\\
ハッピーなちからをくれる\\
運命があなたを守るの\\

トイツの ようです\\
優しい夢の お手伝いするわ\\
飛び立つまでは みな危うい\\
小鳥でいいのよ\\
すこしふらり 自分の道へと\\
続く場所探して 対々和ドラ3\\

出会いって切ない瞬間でしょう\\
繰り返しこころに風を おいてゆく\\

まってて ひとりがふたり\\
ラッキーなつながりはこぶ\\
毎日あなたを見てるわ\\

染め手の ようです\\
とめちゃだめよ 流れない水は\\
濁ってしまうの さあここから\\
旅立つときです\\
ゆれる枝が 自由を導く\\
大丈夫いつでも 面混白中\\

チャンタの ようです\\
優しい夢の お手伝いするわ\\
飛び立つやがて あの大空\\
目指してきれいな\\
軌道えがく 自分の道へと\\
続く場所探して 純全三色
}

\songTitled{ステルス・モ・モ・モード}{東横桃子(斎藤桃子)}
{\albumName}
\songMemo{作詞:\B{畑~亜貴}/作曲・編曲:\B{村井~大}}
\songText{\kasho
どこでも静かに消えれば\\
存在未満のまぼろし\\
私はすべてを見渡す\\
音も声も届かない\\

呼ばれて気付いた\\
自分を求める人が嬉しいっすよ\\

たぶん何もいらなかった\\
でも今はがんばるっす\\
ただついてくだけじゃなくて\\
想いをぶつける\\
次は! そばで待ってるよ\\
次は! ダマと同じリーチ\\
マジに 消えます\\
ステルスモモで 空気より軽い\\

いさかい起きてもいないし\\
観察次第のなりゆき\\
私の能力ムダかも\\
どうせ奇跡起こらない\\

大きく激しく\\
チカラを認めて探してくれたっすよ\\

他人なんていらなかった\\
もしあの日あの場所で\\
またすがたをくらませても\\
きっと降参っすよ\\
隣り! 多分ツモがいいね\\
隣り! 待ちもキレイきっと\\
では 消えましょう\\
マイナス気配 透明の彼方へ\\

たぶん何もいらなかった\\
でも今はがんばるっす\\
ただついてくだけじゃなくて\\
想いをぶつける\\
次で! ドラが重なれば\\
次で! 高めひいてツモ\\
マジに 消えます\\
ステルスモモは 空気さえ自在\\

どこでも静かに消えれば\\
存在未満のまぼろし\\
私はそろそろ沈むよ\\
独りきりが見抜けない\\
音も声も届かない
}

\addcontentsline{toc}{chapter}{\shingo 咲-Saki- THE~夢のヒットスクエア~キャラソン清澄対局編}
\album{咲-Saki- THE~夢のヒットスクエア~キャラソン清澄対局編}

\songTitled{Eternal Wind}{宮永咲(植田佳奈)}
{\albumName}
\songMemo{作詞:\B{林~宏次}/作曲:\B{田代智一}/編曲:\B{安藤高弘}}
\songText{\kasho
ほんと言うとね 自信が持てずに\\
いつも心に 笑顔を隠していた\\
やっと気づいた みんながいるから\\
どこにいたって 素直に笑ってたい\\

緑のじゅうたんに\\
夢を鳴らそう\\
過去よりももっと\\
大切なもの\\

巡りあいが 世界を 動かしてゆく\\
小さな想いだって\\
ほら 花が咲く\\
一人きりじゃ できない ことばかりだよ\\
手を重ね サイを振り\\
回り回ってく 風\\

もっとやれるよ いまならわかるよ\\
ゆずれないもの 自分に負けなくない\\

四角い草原で\\
夢と遊ぼう\\
誰も見たことない\\
一打を打とう\\

本気ならば 世界も 変えられるはず\\
崖の上に咲いた\\
あの 白い花\\
君がいなきゃ できない ことばかりだよ\\
目を閉じて うなづいて\\
勝負かけるから 今\\

巡りあいが 世界を 動かしてゆく\\
小さな想いだって\\
ほら 花が咲く\\
一人きりじゃ できない ことばかりだよ\\
手を重ね サイを振り\\
回り回ってく 風
}

\songTitled{セラフィック・ゲート}{原村和(小清水亜美)}
{\albumName}
\songMemo{作詞:\B{林~宏次}/作曲・編曲:\B{村井~大}}
\songText{\kasho
舞い降りてくる 不思議な気持ち\\
夢色に艶やかに\\
羽ばたいて\\
宇宙を彩る星になる\\

神がかり 無意識に\\
全身で感じてる\\
開いた\\
セラフィック・ゲート\\

ときめいて ジュンチャン レンチャン\\
美しく リーソク サンショク\\
みんなとの心の絆 誰よりも 輝いています\\

きらめいて スーアン タンキ\\
鮮やかに チーホー テンホー\\
天使の声が いま 聞こえてきました\\

とめどないほど 溢れてくるよ\\
嬉しくて楽しくて\\
キミのその\\
存在が強くしてくれる\\

躓いて 傷ついて\\
それでも諦めない\\
約束\\
守り続けたい\\

光速の テンパイ ロンハイ\\
つつましく リューイー スーシー\\
私じゃないような私 特別でスペシャルな力\\

最高の チューレン ポウトウ\\
終わらない オーラス パーレン\\
天使の笑顔 ほら 咲き乱れている\\

ときめいて ジュンチャン レンチャン\\
美しく リーソク サンショク\\
みんなとの心の絆 誰よりも 輝いています\\

きらめいて スーアン タンキ\\
鮮やかに チーホー テンホー\\
天使の声が いま 聞こえてきました
}

\songTitled{タコスぢから=ユメぢから}{片岡優希(釘宮理恵)}
{\albumName}
\songMemo{作詞:\B{畑~亜貴}/作曲・編曲:\B{酒井陽一}}
\songTexts{3}{\kasho
ココロのピンチ カラダのピンチ\\
助けて! タコスさまさまー\\
キモチがアップ チカラがアップ\\
タコスタコス食え おいしいじょ?\\

まさかコレ知らない?\\
なんとコレ経験ない?\\
おしえてあげよーか?\\
おれいはホレ 決まってる!\\

ワタシのゲンキ ホントのゲンキ\\
最強! タコスうまうまー\\
起家でジャンプ 親リージャンプ\\
タコスタコスなら かんたんだぁ?\\

いまやコレじょーしき!\\
だからコレ切らしちゃ めっ!\\
こぼさずいただけよ!\\
もぐもぐホラ たまんないね!\\

来ル来ル来ル いいなき\\
来ル来ル来ル いいツモ\\

チーズおにくレタストマト つつみましょー\\
かたいやわいパリッとしっとり トルティーヤ\\
負けないよって 負けないよって\\
手が腕が鳴る れっつでごー!\\
てんぱいてんぱい そくりー そくりー はじけましょー\\
りーづもりーづもドラドラあったよ サルサソース\\
彩りあざやかだじょ 勝負も楽しくなっちゃう\\
おおきな おくちで わーい!\\

ユウウツにパンチ ナミダにパンチ\\
最悪! タコれふがふがー\\
東場はトップ いつでもトップ\\
タコスぢからだじぇ あんしんだぁ?\\

ひとつトカ足りない!\\
ふたつトカもの足りない!\\
おかわり朝も夜も!\\
ぱくぱくマダ とまんないね?\\

来い来い来い ドラドラ\\
来い来い来い ウラドラ\\

焼いて炒めて生でそのまま はさみましょー\\
からいあまいすっぱいミックス タケリーア\\
毎日だって 毎週だって\\
目が舌が呼ぶ おーるがらいと!\\
ヤオチュウばっかりそろってたら はじめましょー\\
やくまんだってあがってなんぼよ ハラペーニョ\\
充電おわってるじょ いつでもかかってこい\\
でっかい たいどで はーい!\\

いまやコレじょーしき!\\
だからコレ切らしちゃ めっ!\\
こぼさずいただけよ!\\
もぐもぐホラ たまんないね!\\

来ル来ル来ル つもるじょー\\
きっと あるよ あがりハイ\\

来ル来ル来ル いっぱつ\\
来ル来ル来ル ウラドラ\\

チーズおにくレタストマト つつみましょー\\
かたいやわいパリッとしっとり トルティーヤ\\
負けないよって 負けないよって\\
手が腕が鳴る れっつでごー!\\
リーヅモ いっぱつ どらさんどらさん はじけましょー\\
いんぱちいんぱち 親っパネで サルサソース\\
彩りあざやかだじょ 勝負も楽しくなっちゃう\\
おおきな おくちで あじわう タコスは\\
勝利の あじだよ わーい!
}

\songTitled{染めて逆転}{染谷まこ(白石涼子)}
{\albumName}
\songMemo{作詞:\B{林~宏次}/作曲:\B{福本公四郎}/編曲:\B{安岡洋一郎}}
\songText{\kasho
染めアガリ スキあらば狙うで\\
悪いのぉ 高めゲッチュー\\
肝心カナメ 眼鏡外したら\\
本気モード 勝負じゃけえ\\

あきらめんよ 最後まで\\
それがツモを引き寄せるカギ\\
いただき\\

ゲンブツ廃棄で\\
何気にテンパる\\
今に驚け 一発逆転\\

染め手がバレても\\
薄々バレても\\
オリる気配 見せず 背水の ど真ん中\\

先輩の 風を吹かすのは\\
主義として 好かんでの\\

麻雀荘 実家の手伝いも\\
勉強のネタになる\\

クールなフリ していても\\
人一倍萌える燃えてる\\
次期部長\\

チンイツ駄目なら\\
ホンイツドラドラ\\
一人勝ちも\\
データどおり\\

半分バレたら\\
100%バレたら\\
少し躊躇 しても なんとなく イケるじゃろ\\

染め手がバレても\\
薄々バレても\\
あえて迷彩 かけてみようか\\

半分バレたら\\
100%バレたら\\
少し躊躇 しても なんとなく イケるじゃろ
}

\songTitled{「ありがとう」の春はまだ早い}{竹井久(伊藤静)}
{\albumName}
\songMemo{作詞:\B{林~宏次}/作曲:\B{隆~勇人}/編曲:\B{安岡洋一郎}}
{\quad\\\kasho
笑い明かす ことなんてね\\
久しぶりだよ 誰かとこんなに\\
思えばただ 夢を見てた\\
いつかあそこで 打てる自分の姿\\

みんなと出会えた なけなしの奇跡\\
後悔せずに 闘うことが 恩返しだよ\\

「ありがとう」の春は まだ早いから\\
いつか言える 時が来るまで\\
精一杯 笑顔でいよう\\
手加減しないで 最高の麻雀を\\

ネタは豊富 わが部室は\\
居心地良くて 授業も忘れそう\\
思えばただ 物語は\\
ここから全て 始まっていたんだね\\

無邪気さの中に 強さの輝き\\
楽しみながら 闘えるはず さあ立ち向かおう\\

「ありがとう」の春は まだ早いから\\
最終局 終える時まで\\
精一杯 私を出して\\
本気でやりたい 最強の麻雀を\\

「ありがとう」の春は まだ早いから\\
いつか言える 時が来るまで\\
精一杯 笑顔でいよう\\
手加減しないで 最高の麻雀を
}

\songTitled{Ride On The Wave!}{清澄高校麻雀部\\
宮永咲(植田佳奈)、原村和(小清水亜美)、片岡優希(釘宮理恵)、染谷まこ(白石涼子)、竹井久(伊藤静)}
{\albumName}
\songMemo{作詞:\B{林宏次}/作曲・編曲:\B{近藤昭雄}}
\songText{\kasho
ちゃっかりと 着席\\
結局は楽しんだ者勝ち\\
あんな手や こんな手とか\\
考えて 日が暮れる\\

千変万化 (国士無双な)\\
しちゃってみない? (ドキドキ)\\
運の荒波 (プライスレスで)\\
乗りこなせ\\

本日は麻雀日和 (いつでも!)\\
今度こそは単独一位 (させるか!)\\
シーパイの前からね\\
勝負は始まってる\\
誰が? 君が? 僕が? 親だ?\\
さあこれからよ\\

ちゃっかりと 笑顔で\\
結局は楽しんだ者勝ち\\
あんな手や こんな手とか\\
考えて 井戸端会\\

千変万化 (国士無双な)\\
しちゃってみない? (ドキドキ)\\
運の荒波 (プライスレスで)\\
乗りこなせ\\

役満は夢のまた夢 (いやいや)\\
そう思っているならたぶん (できない)\\
感じたままでいこう\\
アレや コレや ソレや ドレが\\
広がるかも\\

やってみたい 麻雀したい\\
動き出す気持ちが大切ね\\
ぜったいは ありえないの\\
ハイテイも おまちどう\\

諸行無常の (数え役満)\\
マジカルゲーム (ウラ乗れ)\\
世知辛いほど (四角い海は)\\
大騒ぎ\\

ちょっとコレ 上がってる?\\
トラブルも時には味ごのみ\\
ちゃっかりと 笑顔で\\
結局は楽しんだ者勝ち\\
あんな手や こんな手とか\\
考えて 井戸端会\\

千変万化 (天下無双の)\\
しちゃったみたい (ひらめき)\\
運の荒波 (キミと一緒に)\\
乗りたいね
}

\addcontentsline{toc}{chapter}{\shingo 咲-Saki- Portable}
\album{}

\songTitled{君が笑ってくれたから、今日も夢の種を蒔こう。-キミたね-}{}
{咲-Saki- Portable オープニングテーマ}
\songMemo{作詞・作曲・編曲・歌:\B{Little Non}}
\songText{\kasho
朝猛ダッシュ くせ毛はぐるり\\
カバンぶちまけ 正門アウト(ファイト!)\\
時計を5分 早めてみても\\
チャイムは残酷(きんこんかんこん)\\
考えるフリ 教科書ノート\\
ナイショ内職 大スペクタクル(うぉう)\\
テスト出るとこ 公式・つづり\\
もっと他のこと(おしえて)\\
放課後に 1人立ち尽くす\\
やることやったら どこに行ける?\\

(Let's be ambitious more)\\
抱きしめた夢 止まらない夢\\
確かめるように 歩く 帰り道\\
変えたくて今 種蒔きながら祈った\\
輝けるように 君と笑って行こう\\

また朝が来た ほら寝坊だし\\
働け頭 目覚めろ身体(いっぱーつ!)\\
思ってるのに ついてこない\\
いいや もう歩こう(でんどんどんでん)\\
お日様ぽかり 緑がきれい\\
誰もいないし サボっちゃおうか(あは☆)\\
ああでも やっぱ\\
やっぱり君と 一緒にいたい(あいたい)\\

教室ドア 前に立ち尽くす\\
勇気を出したら 何ができる?\\

(Let's be ambitious more)\\
逃げ出した夢 追いかける夢\\
もどかしすぎて 階段 段飛ばし\\
蒔いて忘れた 種から芽が出ていたよ\\
育てるように 君に笑って行くよ\\

立て見ろ\\
夢抱け\\
今行け\\
花咲け\\

Future! Watcher!\\
Player! Answer!\\
Fighter! Winner!\\
We are! Dreamer!\\

(Let's be ambitious more)\\
抱きしめた夢 止まらない夢\\
確かめるように 歩く 帰り道\\
変えたくて今 種蒔きながら祈った\\

今を壊して 君を笑わせに行くよ\\
}

\songTitled{爆発始動レボリューション}{Little Non}
{咲-Saki- Portable エンディングテーマ}
\songMemo{作詞・作曲・編曲:\B{Little Non}}
{\kasho
抱きしめろ 今を 君と爆発 レボリューション\\[-.3em]

「平気、大丈夫」だって 当たり前みたいに\\
我慢してばっかじゃ 失くしちゃうんじゃない?\\
奥に隠してきた 気持ちが目移り\\
しないようにもっと 道を決めた\\[-.3em]

本気で かかれば 100倍 君となら 無限だから\\[-.3em]

響け 止まらない地球の隅々まで 僕らのパワー\\
永久に 刻むよ 声が枯れ果てるまで\\
君とのメロディー\\[-.3em]

残して行こうね 君と爆発 レボリューション\\[-.3em]

「たぶん、できない」なんて 決め付けたとして\\
笑顔になるのは 僕らじゃないんじゃない?\\
ほんの少しのこと 簡単に見えても\\
難しい時には 見ててあげる\\[-.3em]

視点 変えれば1000倍 君はほら 無敵だから\\[-.3em]

届け 高まる鼓動の先に見えた 未来のアワー\\
永久に 続くよ 命燃え尽きるまで\\
君との メモリー\\[-.3em]

どんどん増えてく 君が爆発 このドキドキ\\
ギラギラ しちゃうね 君と爆発 レボリューション\\[-.3em]

君に 伝えたい気持ちは確かなのに言葉じゃ 足りないよ\\[-.3em]

響け 止まらない地球の隅々まで 僕らのパワー\\
永久に 刻むよ 声が枯れ果てるまで\\
君とのメロディー\\[-.3em]

変えられる 君と 僕も爆発 レボリューション\\
抱きしめろ 今を 君と爆発 レボリューション
}


\addcontentsline{toc}{chapter}{\shingo 咲-Saki- 阿知賀編 episode of side-A}
\album{MIRACLE RUSH}

\songTitled{MIRACLE RUSH}{StylipS\\
小倉~唯・石原夏織・能登有紗・松永真穂}
{咲-Saki- 阿知賀編 episode of side-A オープニングテーマ}
\songMemo{作詞:\B{こだまさおり}/作曲・編曲:\B{山口朗彦}}
\songText{\kasho
輝いてここ一番 自分の直感を信じて\\
受け取った MIRACLE RUSH\\
いま最高の奇跡に乗り込め\\

冗談じゃない、わかるでしょ(you know?)\\
油断できないこの事情(good luck)\\
究極の選択は永遠のトラウマ\\

勝ち負けじゃないキレイゴト(I know)\\
励ましてくれてアリガト(thank you)\\
最後には自分なんだ 覚悟決めなくちゃ\\

気になるサジェスチョン\\
理屈は時々無力でジレンマ…\\
流されてみてもいいのかも?\\

運命が回りだす 出たトコ勝負ではじめるの\\
心の準備のアルナシは 待ってくれない\\
輝いてここ一番 自分の直感を信じて\\
受け取った MIRACLE RUSH\\
いま最高の奇跡に乗り込め\\

その近道はアリかもね(danger!?)\\
罠ワナしてて燃えちゃうよ(nice choice!)\\
ハードルの感じ方 試されてるみたい\\

そこは手堅く大胆に(of course)\\
遊びゴコロも忘れずに(my way)\\
アンパイじゃつまんないね 割とギャンブラー\\

事態はサディスティック?\\
負ける気はサラサラないの It's all right\\
強気なくらいがちょうどいい!\\

運命が動きだす シンケン勝負で進むんだ\\
わたしらしい上昇志向 大事にしたい\\
心配は後回し リアルな問題が先でしょ\\
いつだってまかせてね\\
ホラ最新のわたしが最強\\[1em]

読めない未来 手探りなら\\
知りたい期待 誘われちゃえ\\
キメたいでしょ 次のチャンス\\
攻めてく姿勢がカンジンだね!\\
My name is きっと Lucky girl, YES? NO? YES!! YES!!\\

運命が回りだす 出たトコ勝負ではじめるの\\
心の準備のアルナシは 待ってくれない\\
輝いてここ一番 自分の直感を信じて\\
受け取った MIRACLE RUSH\\
いま最高の奇跡に乗り込め
}

\album{SquarePanicSerenade}

\songTitled{SquarePanicSerenade}{阿知賀女子麻雀部\\
% \eighthnote
高鴨穏乃(悠木碧)、新子憧(東山奈央)、松実玄(花澤香菜)、松実宥(MAKO)、鷺森灼(内山夕実)}
{咲-Saki- 阿知賀編 episode of side-A エンディングテーマ}
\songMemo{作詞・作曲・編曲:\B{ZAQ}}
\songTexts{3}{\kasho
頂点まであと一息\\
アガリーち All Right!!\\
咲き誇れ\\

東から西まで 北から南まで\\
走り抜ける 四方八方\\
全身全霊\\

握るも捨てるも ドキドキ選択\\
センチメンタルに ドヤッと!\\
また騒ごうよ\\

こっちでも (牌!)\\
あっちでも (牌!!)\\
偶然でも 幸運でも\\
引き寄せて 乗っちゃって\\
せい! a-c-h-i-g-a れっつごー!\\
(まーじゃんぱにっくもういちど Here we Go!)\\

あの舞台を転げ回って\\
奏でちゃってよ、My セレナーデ\\
星屑みたいキラキラって\\
四角い箱の中 夢いっぱい\\

楽しくなってきちゃったね\\
くるくる回る、ターン to ターン\\
駆け引き、勝ち負け、\\あの子に繋いで\\
この草原駆け抜けよう\\
\newcolumn
悔しい涙で~ステージを濡らして\\
優しい笑顔で 抱き止める\\
次は私だ\\

しょっぱなの一打\\ 歯車回りだす\\
自分次第のショータイム\\ 皆はしゃごうよ!\\

有名でも (牌!)\\
無名でも (牌!)\\
関係ない 勝てば官軍だ\\
レジェンド もう一度\\
(穏乃)指切りげんまん! \\(憧)あの子に会いたい\\
(宥)なんだか眠い。。\\(玄)ちょっとお姉ちゃん!\\
(灼)はぁ…。みなさん?いきますよ?\\
(Yeah!!試合開始〜!)\\

願い背負って勝ち進め\\
指先が歌うよセレナーデ\\
掴もう奇跡 ドラドラって\\
積み上がってく数 そう絶対\\

四角の空を 自由にさ\\
飛び回ろうよ、一緒にね\\
笑って 怒って\\ 明日へと繋いで\\
一歩ずつ駆け上ろう\\
\newcolumn\quad\\
また遊ぼうよ \\(Future is bright)\\
ありのままの私たちで \\(Dream come true)\\
1、2、3、4、5のステップを\\
(テンパリぱにっくさいごに\\ Here we go!!)\\

あの舞台を転げ回って\\
奏でちゃってよ、My セレナーデ\\
星屑みたいキラキラって\\
四角い箱の中\\ 夢いっぱい 箱いっぱい\\

まだまだ続くこの勝負\\
走りぬけてく Run the world\\
駆け引き、勝ち負け、\\あの子に繋いで\\
笑って 怒って\\ 明日へと繋いで\\
この草原駆け抜けよう\\

徹底的 射程距離圏内\\
ここ一番 ぞっくぞく!!\\
打ち上がれ
}

\album{Futuristic Player}

\songTitled{Futuristic Player}{橋本みゆき}
{咲-Saki- 阿知賀編 episode of side-A エンディングテーマ}
\songMemo{作詞・作曲・編曲:\B{ZAQ}}
\songText{\kasho
約束の場所へ Over future\\

Image...\\
加速して行く未来に\\
描いてく奇跡の形\\
君と手を繋いでるのに\\
諦めるなんてありえないよ\\

手のひらには\\
戦いあって\\
積み重なった傷跡\\
さぁ、いこう?旅をしよう\\
銀河が残してきた記憶に新たなる一手\\

こぼれ響く音がリズムになって\\ 夢咲かせようと\\
フィールドを歌うわ so futuristic player\\
重なる想い 次の笑顔へ\\
そう、確かな勝利へ\\
エールになるように 指を絡めるの\\
辿りついて Over future\\
\newcolumn
fear...\\
楽しんでるつもりでも\\
いつもある焦燥感に\\
限りない不安覚えながら\\
その中の光信じてるよ\\

手を伸ばせば\\
そこにあるのに\\
届きそうも無い影に\\
「近づいてる?」「まだ遠いの?」\\
自分じゃわからなくて\\
ただ前だけ見つめてる\\

吹き抜けてく風を追い越していく\\ もう誰より速く\\
フィールドに舞い散る so futuristic player\\
必ず進む 明日への音\\
笑い合える日まで\\
共に過ごす時間 裏切りはしない\\
巡りあって Over future\\

こぼれ響く音がリズムになって\\ 夢咲かせようと\\
フィールドを歌うわ so futuristic player\\
重なる想い 次の笑顔へ\\
そう、確かな勝利へ\\
エールになるように 指を絡めるの\\
辿りついて Over future
}

\album{Step One!!}

\songTitled{TSU~・~BA~・~SA}{StylipS\\
小倉~唯・石原夏織・能登有紗・松永真穂}
{咲-Saki- 阿知賀編 episode of side-A オープニングテーマ}
\songMemo{作詞:\B{こだまさおり}/作曲・編曲:\B{山口朗彦}}
{\quad\\[-.2em]\kasho
弱音を口にしたら そうなってしまいそうで\\
臆病な自分を ムリに追い出してみる\\
世界はまだちっちゃくて でも今はここがリアル\\
わたしのすべて\\

明日ばかり見てるワケじゃない\\
成長した実感にちゃんと 気づいてあげなきゃ\\

羽ばたいて “お願い” ぎゅっとチカラをこめるよ\\
夢が夢から 目覚めはじめてる\\
解き放つオモイと 追い風がかさなる場所\\
すぐに見つけにいくから\\
翼 感じてFly あの空へ\\[.8em]

憧れは遠いほうが 安心していられた\\
選ばれる未来 どこか信じきれずに\\
引き返せないくらい 手を伸ばしたその先で\\
やっとわかるの\\

不安定なココロ映した 今日の風が雲間を抜ける\\
動きだした新しい空 深呼吸してひとつになろう\\
透き通る朝に 飛び立つよ\\[.8em]

羽ばたいて “お願い” ぎゅっとチカラをこめるよ\\
夢が夢から 目覚めはじめてる\\
解き放つオモイと 追い風がかさなる場所\\
すぐに見つけにいくから\\

抱きしめて 勇気を わたしだけの輝きで\\
いつも消えない 道しるべにしよう\\
どこまでも広がる 世界がいつかのリアル\\
きっと叶えにいくから\\
翼 感じてFly あの空へ
}

\addcontentsline{toc}{chapter}{\shingo 咲-Saki- 阿知賀編 episode of side-A キャラクターソング}
\album{咲-Saki- 阿知賀編 episode of side-A キャラソンベストアルバム THE 夢のヒットスクエア~阿知賀編}

\def\hoshi{$\!$☆$\!$}

\songTitled{YES!! READY to PLAY}{高鴨穏乃(悠木碧)}
{\albumName}
\songMemo{作詞・作曲・編曲:\B{ZAQ}}
\songText{\kasho
配牌りゃんしゃん\\
七対にこにこ\\
一向聴?まだいーしゃんてん!?\\
できるもんなら役満貫♪\\

駆け足でのぼるくだる坂道\\
鼻歌まじりでご機嫌フィーバー\\ Let's sing a song♪\\
みんなが集まるあの教室へ\\
一番に乗り込むんだ\\

個性って いっちゃえばいろいろある\\
ひとりひとりが 特別で大切だよ\\
そんでもってあたしも主人公\\
もちろん輝いていたい!\\

Yes!!\\
はやく行こうよ またはしゃぎたいんだ\\
でもね焦りは禁物\\
指にマメ作り 明日を攻略!\\
アガる?アガれない?\\
ツモる?ツモれない?\\
わくわくじゃん!大きな花火\\
打ち上げちゃおう\\

遊び感覚じゃ超えられない壁\\
大事なのはラッキーとガッツと集中力ってね♪\\
穏やかなのは名前だけ?\\
あっちとこっち 最高に元気\\

捨てる牌がない\\
アレ?これツモってんじゃない?\\
切るより切り開かなきゃ安牌\\
鳴かぬなら鳴かせてあげようヤオ九牌\\

勝てないって、思った相手にも食らいつく\\
戦いたいのはもっと先にいる\\
あたしのこと忘れてないよね?\\
そこに辿り着くまで絶対…\\

絶対負けないぞ また遊びたいんだ\\
だから強くなるって決めた\\
凄いセンパイが そばにいるしさ!\\
流す?流せない?\\
捨てる?捨てれない?\\
楽しいじゃん!いい試合\\
みんな一緒にさ\\

あの舞台を転げ回って…かな\\

Yes!!\\
はやく行こうよ またはしゃぎたいんだ\\
風が来てるから 鳴こう\\
染め上げるアクティブに 無敵の連鎖!\\
アガる?アガれない?\\
ツモる?ツモれない?\\
わくわくじゃん!真っ向勝負\\

絶対負けないぞ また遊びたいんだ\\
アガる前には深呼吸\\
凄い仲間たち そばにいるから!\\
流す?流せない?\\
捨てる?捨てれない?\\
うきうきだ!素敵な試合\\
咲き乱れよう
}

\songTitled{まじかる\hoshi まーじゃん\hoshi わーるど ver.穏乃}{高鴨穏乃(悠木碧)}
{咲-Saki- 阿知賀編 episode of side-A キャラクターソング Vol.1 高鴨穏乃}
\songMemo{作詞・作曲・編曲:\B{ZAQ・木之下慶行}}
\songTexts{2}{\fontsize{11}{11.5}\kasho
イースーチー\\
リャンウーパー\\
チッチ チッチ\\
クンロクハッセン\\
ピンピンピンロク ピンピンピンロク\\
ニーヨンマルニーヨン=親倍\\
まじヤバいオールで役満!\\

男前に鳴きまくり ハダカ単騎で\\
ぽんぽんちーちー\\
ツモるわきゃ無いよ 捨てておくれよ\\
意地悪だなぁ 隣の住人\\

オタカゼに限って\\
トイツーになっちゃって\\
ポイ捨てしてたら三枚目 はい。\\
なんだコレ?えーっと待って\\
損した?得した?いったいどっちなの!?(数字苦手〜)\\

あがれない… 腕のモンダイ?\\
いやそれよりツキあるかないかでしょ(ありありで)\\
期待にドキドキ 胸がワクワク\\
そして頭ハネ(オーノー!)\\

まじかるまーじゃんわんだーらんど この大きな宇宙\\
大好き 大好き 喜怒哀楽の繰り返し(ポン!)\\
ロジックも直観も天才も凡人も 受け入れてくれるこの空\\
大好き 大好き ツキがある世界\\
読んで 待って 染めて 鳴いて ほら出来た!(ロン!)\\

100点 1000点 5000、10000\\
数えど数えどハコテンぶっ飛び\\
焼き鳥 追い打ち 勘弁してよぉ\\
さいころ コロコロ 山を決めるよ\\

強過ぎ… わかってんだ\\
キャリアの長さは負けた回数さ\\
まさかの配牌 即ダブリーで\\
今日はツイてるな♪(ラッキーパンチ!)\\

まじかるまーじゃんてんぱりわーるど この大きな草原\\
大好き 大好き 百花繚乱女の園(チー!)\\
ロジックも直感も天才も凡人もヤミツキになるこの空\\
大好き 大好き 調子に乗っちゃうよー!\\
先鋒 次鋒 中堅 副将 そして大将だー(いえーい☆)\\

めんたんぴんいっぱつ…どらどら(はねまん〜♪)\\
じゅんちゃんさんしきいーぺーこー(はねまん〜♪)\\
ぴんふりゃんぺいこー…どらどら(はねまん〜♪)\\
りーづもちゅんめんはん…どやどや!\\
はぁ…あったか〜い (こらあー!)\\

絶対負けないぞ また遊びたいんだ\\
だから強くなるって決めたんだ\\

まじかるまーじゃんわんだーらんど この大きな宇宙\\
大好き 大好き 喜怒哀楽の繰り返し(カン!)\\
ロジックも直観も天才も凡人も受け入れてくれるこの空\\
大好き 大好き ツキがある世界\\
読んで 待って 染めて 鳴いて ほら出来た!(ロン!)
}

\songTitled{Live A-Life}{新子憧(東山奈央)}
{\albumName}
\songMemo{作詞・作曲・編曲:\B{ZAQ}}
\songText{\kasho
平常心でいられない\\
心揺さぶる展開\\
だってこの四角い空は\\
だれの天気予報も意味ない\\

いつもスキだらけなんだから\\
勢いだけじゃ渡れないんだよ\\
テンションあげればどうにかなる\\
それだけで勝てる相手じゃない\\

でもあのとき電話くれなきゃ\\
ここへの一歩、踏み出せなかった\\
「まず一人、ここにいる!」って\\
言う勇気 くれたよね\\

新しいLife これからがLive\\
全国を目指すに充分なSoul Five!!\\
きめられたレールに沿った\\
プレイじゃ満足できない\\
難しく考え過ぎて\\
決断力鈍ってるんじゃないの?\\
ほら そこダマでロンだよ!\\
点数読み上げる この勝ちどきが聞こえる?\\
\newcolumn
河を読むずっとじっと\\
捨牌眺めて勝負所\\
計算高いって褒め言葉?\\
ふふっ…油断しないでね\\

食いタン、三色、ドラドラ満貫\\
幾何学的に美しいでしょ?\\
役に呼ばれるようじゃダメ\\
操ることが 究極だ\\

憧れのLife 瞬間をAlive\\
近道せず終局へ I'll dive!\\
涙みせるのは弱くない\\
弱音はかないのが強さじゃない\\
簡単なことなのよ\\
私らしく勝負するだけだ\\
「まさか」の連続でしょ?\\
きっと這い上がる 論破してく無常のパズル\\

Jump 指先で跳ねる\\
Bounce 心躍らせて\\
East South West North\\
流れを乗りこなす\\
もっと上げるボルテージ!\\

新しいLife これからがLive\\
全国を目指すに充分なSoul Five!!\\
きめられたレールに沿った\\
プレイじゃ満足できない\\
難しく考え過ぎて\\
決断力鈍ってるんじゃないの?\\
もう 一気に決め込んじゃうよ\\
点数読み上げる この勝ちどきが聞こえる?
}

\songTitled{まじかる\hoshi まーじゃん\hoshi わーるど ver.憧}{新子憧(東山奈央)}
{咲-Saki- 阿知賀編 episode of side-A キャラクターソング Vol.2 新子憧}
\songMemo{作詞・作曲・編曲:\B{ZAQ・木之下慶行}}
\songTexts{2}{\fontsize{11}{11.5}\kasho
イースーチー\\
リャンウーパー\\
チッチ チッチ\\
クンロクハッセン\\
ピンピンピンロク ピンピンピンロク\\
ニーヨンマルニーヨン=親倍\\
まじヤバいオールで役満!\\

タコ鳴きならぬタコりー\\
引っ掻き回すの 困ったちゃんだわ\\
アガるときは 考えてよね\\
わかってるって? ゲームは知性\\

ちょっと点取って\\
気取ってみちゃって\\
見逃しキメ込んで(うーわー)\\
信じらんない ツキも逃げてく\\
上がってなんぼ ゲームの極意\\

危機一発 チャンス掴む\\
読むのはすきだから(ツッパって)\\
五感がゾクゾク 知的に見抜く\\
スピードが命 ねっ♪\\

まじかるまーじゃんわんだーらんど この大きな宇宙\\
大好き 大好き 喜怒哀楽の繰り返し(ポン!)\\
ロジックも直観も天才も凡人も受け入れてくれるこの空\\
大好き 大好き ツキがある世界\\
読んで 待って 染めて 鳴いて ほら出来た!(ロン!)\\

国士、三槓子、九連宝燈\\
やってみたいけどムリムリムリムリ…\\
どんな確立? ありえないでしょ\\
とにかくアガるわ どんな役でも\\

一期一会 早く来て\\
待ち過ぎは怖いの(流れちゃう)\\
ココしかない ヤバい待ちで\\
あなた狙い打ち\\

まじかるまーじゃんてんぱりわーるど この大きな草原\\
大好き 大好き 百花繚乱女の園(チー!)\\
ロジックも直感も天才も凡人もヤミツキになるこの空\\
大好き 大好き 調子に乗らないで!\\
笑って 泣いて 怒って 大事な 仲間達(いえーい☆)\\

めんたんぴんいっぱつ…どらどら(はねまん〜♪)\\
じゅんちゃんさんしきいーぺーこー(はねまん〜♪)\\
ぴんふりゃんぺいこー…どらどら(はねまん〜♪)\\
りーづもちゅんめんはん…どやどや!\\
はぁ…あったか〜い(こらあー!)\\

新しいLife これからがLive\\
全国を目指すに充分なSoul Five!!\\
きめられたレールに沿った\\
プレイじゃ満足できないから\\

まじかるまーじゃんわんだーらんど この大きな宇宙\\
大好き 大好き 喜怒哀楽の繰り返し(カン!)\\
ロジックも直観も天才も凡人も受け入れてくれるこの空\\
大好き 大好き ツキがある世界\\
読んで 待って 染めて 鳴いて ほら出来た!(ロン!)
}

\songTitled{Dragon Magic}{松実玄(花澤香菜)}
{\albumName}
\songMemo{作詞・作曲・編曲:\B{ZAQ}}
\songText{\kasho
たった一人 守り続けてきた\\
この学び舎の一角\\
またあのころの香りに\\
戻れる気がしているの\\
ドアが開く音 元気におはようって\\
卓をみんなで囲むよ\\
でもあのころより 随分みんな\\
大人になっているよね\\

ファンタスティックな\\
オカルトなのかな\\
ドラマティックな\\
展開がいいかな\\

幸せが、幸運が、集まってくるよ\\
最初はたった一つでも\\
引き寄せてしまう不思議 魔法は続いてく\\

ドラドラ いつでも一緒に居るよ\\
もしくは裏だってのるかも?\\
でもちょっぴり 照れちゃうなぁ\\
ドラゴンロード、なんてね\\

ロマンティックな\\
勝利がいいな\\
ドラスティックも\\
嫌いじゃないかな\\

喜びが、楽しさが、近づいてくるよ\\
私にしかきっとない力\\
信じてるから もっと強くなれるんだ\\

リー ヅモ ドラ ドラ…\\
リー ヅモ ウラ ウラ…\\
ほら、笑顔で挨拶だよ\\

幸せが、幸運が、集まってくるよ\\
最初はたった一人でも\\
みんながいるから 戦いは続いてく\\
喜びが、楽しさが、近づいてくるよ\\
私にしかきっとない力\\
信じてるから もっと強くなれるんだ
}

\songTitled{まじかる\hoshi まーじゃん\hoshi わーるど ver.玄}{松実玄(花澤香菜)}
{咲-Saki- 阿知賀編 episode of side-A キャラクターソング Vol.3 松実玄}
\songMemo{作詞・作曲・編曲:\B{ZAQ・木之下慶行}}
\songTexts{2}{\fontsize{11}{11.5}\kasho
イースーチー\\
リャンウーパー\\
チッチ チッチ\\
クンロクハッセン\\
ピンピンロク ピ…ん?ロク\\
ニーヨンマルニーヨン=親倍\\
まじヤバいオールで役満!\\

先鋒 それ点取り屋\\
エースポジションなんて\\
気が引けるよ\\
頑張るってゆーてもドラ爆\\
タイミングが重要なの…は〜ぁ\\

でも1つドラ捨てちゃうと\\
もう来ない気がして 切り込めないの\\
泣きっ面見せてる場合じゃないね\\
もっと頑張るから…見ててね!\\

トップもラスも引かない\\
存在感ない打ち方なんて\\
振り切って 打ってこそだから\\
3チャもイヤだよ!いやだよぅ\\

まじかるまーじゃんわんだーらんど この大きな宇宙\\
大好き 大好き 喜怒哀楽の繰り返し(ポン!)\\
ロジックも直観も天才も凡人も受け入れてくれるこの空\\
大好き 大好き ツキがある世界\\
読んで 待って 染めて 鳴いて ほら出来た!(ロン!)\\

ホンイツ ドラ4 ウラドラ乗ります\\
えげつないぐらい一発で…ツモ♪\\
ごめんね、平和な牌の正体は\\
罠だったりするかもです…うん!\\

配牌 綺麗でありたい\\
私の手の内は 赤いから\\
ここぞとばかりに 遺憾なく\\
火を吹いちゃうの ふぅー!\\

まじかるまーじゃんてんぱりわーるど この大きな草原\\
大好き 大好き 百花繚乱女の園(チー!)\\
ロジックも直感も天才も凡人もヤミツキになるこの空\\
大好き 大好き 調子はどうですか?\\
積んで 立てて 乗せて まくって 大逆転(いえ〜い☆)\\

めんたんぴんいっぱつ…どらどら(はねまん〜♪)\\
じゅんちゃんさんしきいーぺーこー(はねまん〜♪)\\
ぴんふりゃんぺいこー…どらどら(はねまん〜♪)\\
りーづもちゅんめんはん…どやどや!\\
はぁ…あったか〜い(こらぁー!)\\

幸せが、幸運が、集まってくるよ\\
最初はたった一人でも\\
みんながいるから続いてく\\

まじかるまーじゃんわんだーらんど この大きな宇宙\\
大好き 大好き 喜怒哀楽の繰り返し(カン!)\\
ロジックも直観も天才も凡人も受け入れてくれるこの空\\
大好き 大好き ツキがある世界\\
読んで 待って 染めて 鳴いて ほら出来た!(ロン!)
}

\songTitled{麻雀あったかぽっぷ}{松実宥(MAKO)}
{\albumName}
\songMemo{作詞・作曲・編曲:\B{ZAQ}}
\songText{\kasho
麗らかな日差しが\\
校舎を明るく照らした\\
ぽかぽかで気持ちいいな\\
マフラーに埋めるほっぺた\\
みんなの笑い声\\
引き寄せられる方へ\\
私教室の端っこで\\
暖をとってしあわせ\\

「あれ、もう帰っちゃうの?」\\
「もう一回打ってこうよ」\\
「楽しくて仕方ないの」\\
「私もおんなじだよ」\\
ゆっくり ゆる〜り\\
打つ手は緩めずに\\
でも夜は冷えるから\\
遅くなる前に帰りたいなぁ\\

ひだまりいっぱいの ぶかつびより\\
ゆらゆら ほかほか ふわふわ\\
ツモったら散る花びら\\
落ち着ける赤い牌\\
こんなにあったか〜い\\

こんなに気分のいい日は\\
ホットココアを飲みたいな\\
春夏秋冬 震えながら\\
妹と打つ日々 楽しいな\\
まずは無難に対々和\\
遠慮がちに一盃口\\
フリテン注意で一通も\\
赤でそろえた中混一色\\

筒子、萬子、索子、花も風も\\
どの子にしよう 迷うよね\\
直観、幸運、信じて判断\\

はればれとした空 光に触れて\\
ぴかぴか きらきら そよそよ\\
ツモったら暖房つけていい?\\
リーチで一回休んでいい?\\
春風が気持ちいいから\\

マフラーの中には夢も詰まっているの\\
全自動麻雀卓にいつか\\
こたつ布団を敷きたいなぁ…だめかなぁ?\\

ひだまりいっぱいの ぶかつびより\\
ゆらゆら ほかほか あったか〜い\\
そこ赤信号だよ\\
わたしの光 差し込んじゃう\\
暖かい赤い牌 大好きなの
}

\songTitled{まじかる\hoshi まーじゃん\hoshi わーるど ver.宥}{松実宥(MAKO)}
{咲-Saki- 阿知賀編 episode of side-A キャラクターソング Vol.4 松実宥}
\songMemo{作詞・作曲・編曲:\B{ZAQ・木之下慶行}}
\songTexts{2}{\fontsize{11}{11.5}\kasho
イースーチー\\
リャンウーパー\\
チッチ チッチ\\
クンロクハッセン\\
ピンピンロク ピンピンロク\\
ニーヨンマルニーヨン=親倍\\
まじヤバいオールで役満!\\

闇で多面張\\→待ちが増えてった\\
→わからなくなった\\→ムードでバレた…フリテン…。\\

徹夜で麻雀 集中力切れちゃう\\
でも意外と ドラがツいて\\
振り込むと痛いよ。倍満だぁ。\\

赤い牌がスキ\\
温度が伝ってくるの\\
ええと…あるとすれば\\
紅孔雀あたりでどうかなぁ?\\

まじかるまーじゃんわんだーらんど この大きな宇宙\\
大好き 大好き 喜怒哀楽の繰り返し(ポン!)\\
ロジックも直観も天才も凡人も受け入れてくれるこの空\\
大好き 大好き ツキがある世界\\
読んで 待って 染めて 鳴いて ほら出来た!(ロン!)\\

対面が切った\\
牌をポンしてみたいなぁ。\\
どんな強打手だって退かない\\
だってお姉ちゃんだもん\\

青い牌は苦手\\
温度が逃げてくの\\
春の陽気に\\
マフラーなびかせるよ\\

まじかるまーじゃんてんぱりわーるど この大きな草原\\
大好き 大好き 百花繚乱女の園(チー!)\\
ロジックも直感も天才も凡人もヤミツキになるこの空\\
大好き 大好き 調子はイイ感じ\\
悔しい 悲しい 涙も 全部 抱きしめるよ(わぁ〜い)\\

めんたんぴんいっぱつ…どらどら(はねまん〜♪)\\
じゅんちゃんさんしきいーぺーこー(はねまん〜♪)\\
ぴんふりゃんぺいこー…どらどら(はねまん〜♪)\\
りーづもちゅんめんはん…どやどや?\\
はぁ…あったか〜い\\

ひだまりいっぱいの ぶかつびより\\
ゆらゆら ほかほか ふわふわ\\
ツモったら散る花びら\\
落ち着ける赤い牌…あったか〜い\\

まじかるまーじゃんわんだーらんど この大きな宇宙\\
大好き 大好き 喜怒哀楽の繰り返し(カン!)\\
ロジックも直観も天才も凡人も受け入れてくれるこの空\\
大好き 大好き ツキがある世界\\
読んで 待って 染めて 鳴いて ほら出来た!(ロン!)
}

\songTitled{Next Legend}{鷺森灼(内山夕実)}
{\albumName}
\songMemo{作詞・作曲・編曲:\B{ZAQ}}
\songText{\kasho
心の奥に秘めた 熱いもの\\
あなたがいたから 始めたこと\\
あなたがいてまた 始めるもの\\
A legend of my memory...\\

子供の頃 日曜から日曜まで\\
大人に混じり 配牌を見てた\\
おとなしく見えても 甘いわけじゃなくて\\
切り出しヌルいと 容赦できないよ\\

だけどあの人の前では\\
無防備になってしまうよ\\
このネクタイは宝物\\
ずっとなくさない\\

Messenger あなたの\\
背中から伝わるSupervision\\
言葉じゃないプレイで\\
魅せられる 伝説を繋ぐ\\
Voices 感じる\\
この部屋に響く Fanfare\\
三元牌抱いて この胸の高鳴り\\
わたしたちのもの\\
\newcolumn
触れようとも 追いつけない?\\
憧れがあったら 追い越せない?\\
流れてく雲間に映す背中\\
次の世代 築いていく\\

大人ぶったような香りで\\
悟られず さりげなく\\
進めていくのが本物\\
手前であがらせて\\

Messenger あなたの\\
背中から伝わるSupervision\\
言葉じゃないプレイで\\
魅せられる 伝説を繋ぐ\\
Voices 感じる\\
この部屋に響くFanfare\\
三元牌抱いて この胸の高鳴り\\
わたしたちのもの\\

今なら、楽しいと思えるよ。
}

\songTitled{まじかる\hoshi まーじゃん\hoshi わーるど ver.灼}{鷺森灼(内山夕実)}
{咲-Saki- 阿知賀編 episode of side-A キャラクターソング Vol.5 鷺森灼}
\songMemo{作詞・作曲・編曲:\B{ZAQ・木之下慶行}}
\songTexts{2}{\fontsize{11}{11.5}\kasho
イースーチー\\
リャンウーパー\\
チッチ チッチ\\
クンロクハッセン\\
ピンピンロク ピンピンロク\\
ニーヨンマルニーヨン=親倍\\
まじヤバいオールで役満\\

ツイてない\\
十三不塔の配牌\\
ダブロン ターハイ 親ちょんなんて\\
泣きッ面にハチ\\

どうしてさ\\
わたしがリーダーである理由って?\\
落ち着き ないない うちのチーム\\
しっかりしなきゃだ 私が\\

とんでもない役で\\
アガり続ける 強者集まる\\
怖いけど 怖くない\\
負けてられない それは本当\\

まじかるまーじゃんわんだーらんど この大きな宇宙\\
大好き 大好き 喜怒哀楽の繰り返し(ポン!)\\
ロジックも直観も天才も凡人も受け入れてくれるこの空\\
大好き 大好き ツキがある世界\\
読んで 待って 染めて 鳴いて ほら出来た(ロン。)\\

染め手だよ\\
この配牌なら 絶対に\\
マンズが余って 切るとくる\\
来続けるのは何故…なんで…。\\

グローブ装着\\
切り開いてく 勝利の階段\\
リアルな一面 見せるのは今\\
勝ちにいかなきゃ\\

まじかるまーじゃんてんぱりわーるど この大きな草原\\
大好き 大好き 百花繚乱女の園(チー!)\\
ロジックも直感も天才も凡人もヤミツキになるこの空\\
大好き 大好き 調子に乗ってるの?\\
騒いで 遊んで 真面目な 楽しみ 教えられた。\\

めんたんぴんいっぱつ…どらどら(はねまん〜♪)\\
じゅんちゃんさんしきいーぺーこー(はねまん〜♪)\\
ぴんふりゃんぺいこー…どらどら(はねまん〜♪)\\
りーづもちゅんめんはん…どやどや!\\
はぁ…あったか〜い (こらあー!)\\

Messenger\\
あなたの背中から伝わるSupervision\\
言葉じゃないプレイで 伝説を繋ぐの\\

まじかるまーじゃんわんだーらんど この大きな宇宙\\
大好き 大好き 喜怒哀楽の繰り返し(カン!)\\
ロジックも直観も天才も凡人も受け入れてくれるこの空\\
大好き 大好き ツキがある世界\\
読んで 待って 染めて 鳴いて ほら出来た(ロン。)
}

\songTitled{One Vision}{園城寺怜(小倉唯)}
{\albumName}
\songMemo{作詞・作曲・編曲:\B{ZAQ}}
\songText{\kasho
睫毛を揺らして\\
目が覚めた午前9時\\
窓に色彩が降り積もる\\
そっか 今日は昼からやな\\

私の名前呼ぶ\\
仲間がたくさん増えた\\
ありがとうを伝えたいな\\
この眼 ぎゅっと瞑った\\

花が咲く 風が吹いて あなたは鳴くんだろう\\
解ける運命なら知っているよ\\

触れられないはずの未来を覗き見て\\
それは狙い打てるFortune Vision\\
みんながくれた時の歯車まわして\\
変わる 変えていくFortune Vision\\
まだ…もう少し\\
もっと…はやく\\
夢に会えるなら\\

3000 6000…オール\\
リードしてる点棒\\
追われ続けるビハインド\\
ここで 突き放しときたい\\

認めてくれる\\
仲間がたくさん増えた\\
膝枕 気持ちええな\\
この眼 そっと閉じた\\
\newcolumn
立ちふさがる絶対王者 選び打つ道は無限\\
軌道をズラして乗り越える 負けない\\

刹那の輝きを未来に託したい\\
危険な賭けになるかなTaboo Vision\\
この力もって時の心拍数あげてく\\
勝てる そのためのTaboo Vision\\

まだ…もう少し\\
もっと…はやく\\
夢が咲くのなら\\
少しの痛みは怖くない\\

触れられないはずの未来を覗き見て\\
それは狙い打てるFuture Vision\\
みんながくれた時の歯車まわして\\
変わる 変えていくFuture Vision
}

\songTitled{東南西北\hoshi うちだおれ\hoshi わーるど ver.怜}{園城寺怜(小倉唯)}
{咲-Saki- 阿知賀編 episode of side-A キャラクターソング Vol.6 園城寺怜}
\songMemo{作詞・作曲・編曲:\B{ZAQ}}
\songTexts{2}{\kasho
深呼吸してリラックス\\
昔の記憶をリヴィジョン\\
生死の境をさまよって\\
未来引き連れてリバイバル\\

コンビ打ちもお手のモン\\
閃いて呼ぶ運の波\\
単独ブッチギリはさせへん\\
他家にサインして一コロ\\

立直一発 リー棒は垂直\\
宣言したら 皆ひるむ\\
一巡後ろは ウチがトップや\\
どや?\\

打ちだおれ 打ちだおれ\\
ここ大阪から 一発決めたる\\
半チャン終わり まだ次!次!\\
よってらっしゃいな オーラス貫徹\\
負けてられへん 役も喰らわば牌まで\\

先鋒組 点棒取り合い\\
性格と打ち筋は似とる\\
我慢も強がりも見えとるで\\
ウチの目ぇは誤魔化せない\\

でもクラクラ 頭に稲妻\\
あかんここで倒れたら\\
ヤミテンで振込みはせえへん\\
点差縮めて 借りは返す\\

団体戦こそ腕の見せどこ\\
みんなと掴む一巡先\\
気持ちに応えたい\\
見抜いてみせる\\
見えた!\\

打ちだおれ 打ちだおれ\\
直撃一閃 振り込んだれや\\
麻雀セオリー そんなのナイ\\
よってらっしゃいな 役満あがり\\
狙えるかもね ツキは来てるよ続けよう\\[1em]

シングル…ダブル…トリプ…\\
…大丈夫やで\\
牌の重さは気持ちの大きさ\\
無茶してでも欲しいねんな\\
アクセル踏むで…!\\

打ちだおれ 打ちだおれ\\
ここ大阪から 一発決めたる\\
半チャン終わり まだ次!次!\\
よってらっしゃいな もっかい勝負\\
打ちだおれ 打ちだおれ\\
直撃一閃 振り込んだれや\\
麻雀セオリー そんなのナイ\\
よってらっしゃいな 刹那の視界\\
未来攫って “まーじゃんわーるど”行きましょう
}

\songTitled{Little Pray}{清水谷竜華(石原夏織)}
{\albumName}
\songMemo{作詞・作曲・編曲:\B{ZAQ}}
\songText{\kasho
空に憧れた小さな羽\\
飛べるように祈りながら\\

いつも太陽を仰ぐ掌\\
握る四角い嵐に\\
呑み込まれそう\\

「もうやめて」\\
何度も思った\\
でも止められない夢が\\
あることも知っているから\\

大切な輝きは 私が守ろう\\
弱さを翳さないで\\
きっと立てるはず…そうでしょ?\\
安らぎは未来への\\
息吹となるはずだから\\
瞳を塞ぐなら\\
ここに帰ってきて\\
もう傷付かないで\\

圧倒されそうな空気の中で\\
繋げられた足跡に\\
最後の鐘を\\

「負けたくない」\\
何度も漂う\\
そう、5つの祈りを感じて\\
勝利掴みたい\\
\newcolumn
儚くて強いのは 生きてる証\\
その胸に秘められた\\
優しさ 信じているよ\\
倒れそうなときだって\\
私に今出来ることを\\
探して応えるよ\\
だから走ってきて\\
全てを抱きしめる\\

空に憧れた小さな羽\\
飛べるように祈ってるよ\\

「つらくない?」\\
何度も確かめた\\
その度に笑うその顔が\\
愛おしくって\\

大切な輝きは 私が守ろう\\
弱さを翳さないで\\
きっと立てるはず…そうでしょ?\\
安らぎは未来への\\
息吹となるはずだから\\
瞳を塞ぐなら\\
ここに帰ってきて\\
もう傷付かないで
}

\songTitled{東南西北\hoshi うちだおれ\hoshi わーるど ver.竜華}{清水谷竜華(石原夏織)}
{咲-Saki- 阿知賀編 episode of side-A キャラクターソング Vol.7 清水谷竜華}
\songMemo{作詞・作曲・編曲:\B{ZAQ・木之下慶行}}
\songTexts{2}{\kasho
Hey 今日は\\
打たせてもらうで やるきじゅーぶん\\
調子よさげで\\
手牌もなかなか ハイテンション\\

スピード性も重要やな\\
プレッシャーに耐えながら\\
瞬間に答え出す\\
長考の末の見極め\\

テンパイしたら即リーしがち\\
ハンパな読み あかんやつや\\
待ちを広く ジッと両面待ち\\
状況応じて 和了!どや?\\

打ちだおれ 打ちだおれ\\
ここ大阪から 一発決めたる\\
半チャン終わり まだ次!次!\\
よってらっしゃいな オーラス貫徹\\
負けてられへん 役も喰らわば牌まで\\

Hey 今夜は\\
寝かさ〜んで? 徹マン決定\\
箱の中には残り何点?\\
底みて嘆いて\\

リーチ棒すらないなんて\\
鳴き手で安上がり…\\
理論も運も通じない\\
楽しんでなんぼの この世界\\

テンパってる子からブーイング\\
追い込みリーチで超必殺\\
駆け引きのタイミング次第\\
136牌の行方は??\\

打ちだおれ 打ちだおれ\\
直撃一閃 振り込んだれや\\
麻雀セオリー そんなのナイ\\
よってらっしゃいな 役満あがり\\
狙えるかもね ツキは来てるよ続けよう\\[1em]

みんなの期待背負って 部長になるのは\\
怖かったりもした でも\\
関西最強だけじゃ嫌\\
みんなと一緒に…!\\

打ちだおれ 打ちだおれ\\
ここ大阪から 一発決めたる\\
半チャン終わり まだ次!次!\\
よってらっしゃいな もっかい勝負\\
打ちだおれ 打ちだおれ\\
直撃一閃 振り込んだれや\\
麻雀セオリー そんなのナイ\\
よってらっしゃいな 小さな祈り\\
胸にしまって “まーじゃんわーるど”行きましょう
}

\addcontentsline{toc}{chapter}{\shingo 咲-Saki- 阿知賀編 episode of side-A Portable}
\album{}

\songTitled{moment of glory}{橋本みゆき}
{咲-Saki- 阿知賀編 episode of side-A Portable オープニングテーマ}
\songMemo{作詞・作曲・編曲:\B{ZAQ}}
\songText{\kasho
刹那の輝きを つかみ取ろうよ\\

夢の始まりは誰にでもあって\\
その中で諦めなかったから\\
まだここにいる\\
“本気”たちが残って\\
ひとつになった\\
願いがこみ上げる\\
強さの源だね\\

現状維持なんて柄じゃない!\\
目指した光は遠いじゃない?\\
焦らずに走り出した\\
道しるべより自分を信じる\\
誰にも負けられない\\

Precious one!!\\
楽しんで過ごせる今を\\
心から愛してるから\\
恐れは 笑顔で塗り替えていける\\
Feel it now!!\\
チャンスを感じて遊ぶよ\\
私は一人じゃないわ\\
大事な一瞬を (moment of glory)\\
迎えに行こう\\

迷いながら知った道があること\\
いつまでも同じ視点では\\
いられないからね\\
バトン受けとったら\\
期待を背負う\\
近道 危なくて\\
何度も目を瞑る\\

運命か偶然かわからない\\
それでも攻めてなきゃ変わらない\\
現状打破のシンパシー\\
思い出は今 胸にしまった\\
誰にも止められない\\

Let do it!!\\
積み上げた奇跡の数は\\
最後に私が叫ぶ\\
逃げない 絶対に\\
勝負はこれから\\
No more cry!!\\
夢に幾度も泣かされた\\
でもそれは無駄じゃないわ\\
時間を置き去りに\\
(moment of glory)\\
進歩してゆく\\

散るときは儚すぎるね\\
でも一瞬の、一瞬の花が綺麗で\\
私も、私たちも\\
咲かせたいの、今\\

楽しんで過ごせる今を\\
心から愛してるから\\
恐れは 笑顔で塗り替えていける\\
Feel it now!!\\
チャンスを感じて遊ぶよ\\
私は一人じゃないわ\\
大事な一瞬を (moment of glory)\\
迎えに行こう
}


\addcontentsline{toc}{chapter}{\shingo 咲-Saki- 全国編}
\album{New SPARKS!}

\songTitled{New SPARKS!}{橋本みゆき}
{咲-Saki- 全国編~オープニングテーマ}
\songMemo{作詞:\B{畑~亜貴}/作曲:\B{fandelmale(Arte Refact)}/編曲:\B{酒井拓也(Arte Refact)}}
\songText{\kasho
もう一歩踏み出せる?\\
私待ってたよ\\
絶対ゆずれない この時を待ってたよ\\

(New SPARKS! 輝きの中で)\\
(New SPARKS! 違う世界へ)\\

そう誰もこの先わからない\\
だから自分を信じよう\\
トラブルも乗り越えて\\

Ah! 君とここで会えるのも\\
勇気がくれた 素敵な運命なんだね\\

やっと声に出して言えそうだよ\\
「負けたくない」「あきらめない」\\
本音で勝負!\\

夢に何度も TRY! TRY!! 進まなくちゃね\\
新しい風に誘われて 新しい場所が見えた\\
もう一歩踏み出せる?\\
私待ってたよ\\
絶対ゆずれない この時を待ってたよ\\
君とCHANCE! CHANCE!! つかもう\\

(New SPARKS! 輝きの中で)\\
(New SPARKS! 違う世界へ)\\

さあ行くよ 明日の私が\\
今日の私を呼んでるよ\\
すごい事が起こるかも\\

Ah! 君も同じ気持ちかな\\
準備は完了 一緒の景色見ようよ\\

ずっと憧れだと思ってたら\\
「叶えたいな」「やりたいな」\\
本気の笑顔!\\

夢を持ったら CRY! CRY!!\\
熱くなっちゃう\\
素晴らしい仲間誘い出す\\
素晴らしい場所があるね\\
もう一歩踏み出そう!\\
とまらない鼓動\\
そんな始まりの鼓動はとまらない\\

夢を持ったら CRY! CRY!!\\
熱くなっちゃう\\
素晴らしい仲間集まれば\\
熱いよ熱いよ(始まりだよね)\\
とまらないままでいいかな\\
(鼓動きっととまらない)\\

夢に何度も TRY! TRY!! 進まなくちゃね\\
新しい風に誘われて 新しい場所が見えた\\
もう一歩踏み出せる?\\
私待ってたよ\\
絶対ゆずれない この時を待ってたよ\\
君とCHANCE! CHANCE!! つかもう\\

(New SPARKS! 輝きの中で)\\
(New SPARKS! 違う世界へ)\\
}

\songTitled{TRUE GATE}{橋本みゆき}
{咲-Saki- 全国編~エンディングテーマ}
\songMemo{作詞:\B{畑~亜貴}/作曲・編曲:\B{酒井拓也(Arte Refact)}}
\songText{\kasho
(明日への扉 開きかけてるよ\\
明日への道ともに走るよ\\ だから One more chance!)\\

逃れたいのは 自分の痛みが\\
君に伝わりそうで怖いからさ\\
忘れたいのは 傷つけた瞳\\
君の心はまだ熱いままで\\

次の希望があると(きっときっと大丈夫)\\微笑むから\\
最後の扉が開く(希望への扉)\\
(信じてるこの奇跡を)\\

優しいだけじゃない\\
冷たい風が未来を呼ぶ\\
走り出せ 君とだから出来ることなのさ\\
消えないうちに 夢の夢の 夢のなかへ\\
いつまでもあきらめないで\\
(闇に負けないで)あの約束を\\

(孤独へと変わる 閉ざされた嘆き\\
孤独でもまた夜明けがくるよ\\ だから No more pain!)\\

抱きしめさせて 思い出の色に\\
君が流されそうで止めたいのさ\\
抱きしめさせて 偽りを嫌う\\
君の心だけは守りたいよ\\
\newcolumn
祈り続ける日々に(もっともっと大胆に)\\別れを告げ\\
孤独を自ら壊す(新しい場所へ)\\
(確かなものは何だろう?)\\

激しい稲妻\\
それが今の僕らの鼓動\\
走り出せ 君とならばどこまでも行くよ\\
消せない光 夢を夢を 夢を見せて\\
その涙無駄にしないで\\
(ひとり泣かないで)扉を叩け!\\

次の希望があると(きっときっと大丈夫)\\微笑むから\\
最後の扉が開く(希望への扉)\\

優しいだけじゃない\\
冷たい風が未来を呼ぶ\\
走り出せ 君とだから出来ることなのさ\\
消えないうちに 夢の夢の 夢のなかへ\\
いつまでもあきらめないで\\

激しい稲妻\\
それが今の僕らの鼓動\\
走り出せ 君とならばどこまでも行くよ\\
消せない光 夢を夢を 夢を見せて\\
その涙無駄にしないで\\
(ひとり泣かないで)扉を叩け!\\

(明日への扉 開きかけてるよ\\
明日への道ともに走るよ\\ だから One more chance!)
}

\album{この手が奇跡を選んでる}

\songTitled{この手が奇跡を選んでる~宮守女子高校 ver.}{宮守女子高校\\
小瀬川白望(長妻樹里)、エイスリン・ウィッシュアート(水野マリコ)、鹿倉胡桃(豊田萌絵)、臼沢塞(佐藤利奈)、姉帯豊音(内田真礼)}
{咲-Saki- 全国編~エンディングテーマ}
\songMemo{作詞・作曲・編曲:\B{ZAQ}}
\songText{\kasho
選ぶ愛はヒトツ 奇跡なんか待ってたってさ\\
手を伸ばしていざカイホー\\
東南西北そろそろ咲こう\\
ホーラ! あそぼっ\\

何切る? どう読む? 任してっオーラス\\
速攻? 待ち待ち? 支配者だあれ?\\
点棒ぐるぐる ポン!チー!カン!ロン!\\
(麻雀って楽しいよね!)\\

ああ これって 人生の縮図だね\\
あのとき あの場所で あれ捨ててなければ\\
あのとき あの場所で あれ啼いてなければ\\
違う結果があったかも? (ナシナシ!)\\

満貫テンパリ リーチ振り込んで\\
トイメンにやり あぁ乗ってないっ\\
持ち腐れた宝に 逃げちゃう点数\\
四角い宇宙に「もしかして」はないっ\\
(キラリ直感ドキツモいやホイ)\\

底抜けポジティヴで宇宙を走るよ\\
予想を越えてくる 君に会いにきた (牌を感じて)\\
上がらずにいられないほど まっしぐらだよ\\
撃ち落としたい (みんな勝ちたい)\\
勝負で語ろう雀! 雀!\\
All for square space stars\\

いざ 崇高なノミ手で 駆逐しましょ\\
美しく麗らかに 上手く流れ読まなきゃ\\
嘘にうろたえず 勝負ドコロ見極める\\
テンパイまではエントランス (うてうて!)\\
\newcolumn
何回やってもマクられツカない\\
今度こそはと もう10回目\\
調子良くても 結局 最後はトントン\\
サイコロふるえて宇宙に弾ける\\
(あの手その手で半チャンもう一回!)\\

アガッて繋げるよ最高の仲間に\\
悔しさも期待も おまかせあれって (だいじょうぶだよ)\\
楽しんだらそのぶん 牌は応えるはず\\
前をむいて (一緒にないて)\\
扉を開けよう雀! 雀!\\
All for square space stars\\

選ぶ愛はひとつ 奇跡なんか待たないわ\\
想像なんて意味ないし\\
迷って決める一本道こそ\\
力になるって この手は知っている\\
(ちゃっかりしっかり集まり万点!)\\

底抜けポジティヴで宇宙を走るよ\\
予想を越えてくる 君が欲しいんだ (もっと感じて)\\
上がらずにいられないほど まっしぐらだよ\\
撃ち落としたい (みんな勝ちたい)\\
勝負で語ろう雀! 雀!\\
All for square space stars\\

荒らして戦場! 通らば追いリー!\\
イメトレ一発赤ドラのってけっ\\
攻めるか逃げるか? 打点が親バリ\\
決めるよ全国 遊びましょ
}
\songTitled{この手が奇跡を選んでる~永水女子高校 ver.}{永水女子高校\\
\fontsize{10}{12}神代小蒔(早見沙織)、狩宿巴(赤﨑千夏)、滝見春(水橋かおり)、薄墨初美(辻あゆみ)、石戸霞(大原さやか)}
{咲-Saki- 全国編~エンディングテーマ}
\songMemo{作詞・作曲・編曲:\B{ZAQ}}
\songText{\kasho
選ぶ愛はヒトツ 奇跡なんか待ってたってさ\\
手を伸ばしていざカイホー\\
東南西北そろそろ咲こう\\
ホーラ! あそぼっ\\

めんぜんめんぜん 沈黙テンパイ\\
迷彩迷彩 ひっかけテンパイ\\
8000 4000 倍満倍満\\
(麻雀って怖いよね!)\\

ああ これって 人生の縮図だね\\
あのとき あの場所で あれ捨ててなければ\\
あのとき あの場所で あれ啼いてなければ\\
違う結果があったかも? (ナシナシ!)\\

めんぜんちんいつ(面前清一色) 待ちが複雑ね\\
見逃しそうよ フリテンはやだ\\
だからリーチはかけないの 恥ずかしいから\\
四角い宇宙に「完璧」はないの\\
(スラリ通した 根性でツモ!)\\

底抜けポジティヴで宇宙を走るよ\\
予想を越えてくる 君に会いにきた (牌を感じて)\\
上がらずにいられないほど まっしぐらだよ\\
撃ち落としたい (みんな勝ちたい)\\
勝負で語ろう雀! 雀!\\
All for square space stars\\

いざ 崇高なノミ手で 駆逐しましょ\\
美しく麗らかに 上手く流れ読まなきゃ\\
嘘にうろたえず 勝負ドコロ見極める\\
テンパイまではエントランス (うてうて!)\\
\newcolumn
じゅんちゃんさんしき (純チャン三色) 完璧すぎる\\
上がらなくっちゃ この手見せたい\\
だからきょどって バレちゃう ハートが弱いな\\
ラス牌つもって宇宙が広がる\\
(おまけに裏ドラ倍満だ〜)\\

アガッて繋げるよ最高の仲間に\\
悔しさも期待も おまかせあれって (だいじょうぶだよ)\\
楽しんだらそのぶん 牌は応えるはず\\
前をむいて (一緒にないて)\\
扉を開けよう雀! 雀!\\
All for square space stars\\

選ぶ愛はひとつ 奇跡なんか待たないわ\\
想像なんて意味ないし\\
迷って決める一本道こそ\\
力になるって この手は知っている\\
(ちゃっかりしっかり集まり万点!)\\

底抜けポジティヴで宇宙を走るよ\\
予想を越えてくる 君が欲しいんだ (もっと感じて)\\
上がらずにいられないほど まっしぐらだよ\\
撃ち落としたい (みんな勝ちたい)\\
勝負で語ろう雀! 雀!\\
All for square space stars\\

荒らして戦場! ぶっとび覚悟で!\\
テンパイ即リー 一発つもつも\\
のみてもリーチで 最大8000\\
決めるよ全国 遊びましょ
}
\songTitled{この手が奇跡を選んでる~姫松高校 ver.}{姫松高校\\
上重漫(伊達朱里紗)、真瀬由子(佳村はるか)、愛宕洋榎(松田颯水)、愛宕絹恵(中津真莉子)、末原恭子(寿美菜子)}
{咲-Saki- 全国編~エンディングテーマ}
\songMemo{作詞・作曲・編曲:\B{ZAQ}}
\songText{\kasho
選ぶ愛はヒトツ 奇跡なんか待ってたってさ\\
手を伸ばしていざカイホー\\
東南西北そろそろ咲こう\\
ホーラ! あそぼっ\\

鳴いて鳴いてけ タンオヤドラドラ\\
みえみえホンイツ どっちが早いか\\
だまってテンパイ つものみつものみ\\
(麻雀って難しいね!)\\

ああ これって 人生の縮図だね\\
あのとき あの場所で あれ捨ててなければ\\
あのとき あの場所で あれ啼いてなければ\\
違う結果があったかも? (ナシナシ!)\\

親リー かかって みんなおりおりで\\
手持ちのドラが もったいないね\\
うまく回してテンパイ ここは勝負なの?\\
四角い宇宙に「努力賞」はないっ\\
(すすめ根性 絶対ロンロン)\\

底抜けポジティヴで宇宙を走るよ\\
予想を越えてくる 君に会いにきた (牌を感じて)\\
上がらずにいられないほど まっしぐらだよ\\
撃ち落としたい (みんな勝ちたい)\\
勝負で語ろう雀! 雀!\\
All for square space stars\\

いざ 崇高なノミ手で 駆逐しましょ\\
美しく麗らかに 上手く流れ読まなきゃ\\
嘘にうろたえず 勝負ドコロ見極める\\
テンパイまではエントランス (うてうて!)\\
\newcolumn
トップ目狙うは 直撃しかない\\
つもっちゃったら どうしようかな\\
\ruby{二着}{にちゃ}で終わるもいいかな ここは一息ね\\
点棒数えて 宇宙も ほっこり\\
(次の半荘 狙うぜトップ!)\\

アガッて繋げるよ最高の仲間に\\
悔しさも期待も おまかせあれって (だいじょうぶだよ)\\
楽しんだらそのぶん 牌は応えるはず\\
前をむいて (一緒にないて)\\
扉を開けよう雀! 雀!\\
All for square space stars\\

選ぶ愛はひとつ 奇跡なんか待たないわ\\
想像なんて意味ないし\\
迷って決める一本道こそ\\
力になるって この手は知っている\\
(ちゃっかりしっかり集まり万点!)\\

底抜けポジティヴで宇宙を走るよ\\
予想を越えてくる 君が欲しいんだ (もっと感じて)\\
上がらずにいられないほど まっしぐらだよ\\
撃ち落としたい (みんな勝ちたい)\\
勝負で語ろう雀! 雀!\\
All for square space stars\\

荒らして戦場! 役満テンパイ\\
どっからでるのか 自分に来るのか\\
お願いお願い 上がるよ神様\\
決めるよ全国 遊びましょ
}

\addcontentsline{toc}{chapter}{\shingo 咲-Saki- 全国編 Portable}
\album{}

\songTitled{UPDATE:SHINE}{橋本みゆき}
{咲-Saki- 全国編 Portable オープニングテーマ}
\songMemo{作詞:\B{松井洋平}/作曲・編曲:\B{酒井陽一}}
\songText{\kasho

\vfill\mincho
(未発売)
}


\addcontentsline{toc}{chapter}{\shingo 咲日和 ANIMATION}
\album{咲日和 ANIMATION}

% StylipS倒闭了多少次了
\songTitled{ドラマティック*サイクル}{StylipS}
{咲日和~オープニングテーマ}
\songMemo{作詞:\B{真崎エリカ}/作曲・編曲:\B{高田暁}}
\songText{\kasho
% ずっと見上げた
% 空の色が
% いつもより少し
% 昼に見えたよ
% 信号の町の
% 交差点一人
% なんかちょっと誰がに会いたい
% そんな気がした
%
% とびきり
% 特別じゃない関係
% ほんとは
% とって大きなんた
%
% 大好きっていうか
% 空気に似てるね
% きっと一緒だよね
% 曖昧なはずなのに
% 刺激的な
% 当たり前の毎日

\vfill\mincho
(未発売)
}

\songTitled{ころもびより}{龍門渕高校麻雀部\\
龍門渕透華(茅原実里)、天江衣(福原香織)、国広一(清水愛)、井上純(甲斐田裕子)、沢村智紀(大橋歩夕)}
{咲日和~エンディングテーマ}
\songMemo{作詞・作曲・編曲:\B{rino}}
{\kasho\quad
% 我がともがらよ、祝えよ、歌えよ\\
% いざ声あげん、歓喜の声を\\
% 天江衣の生誕の歓びを\\
%
% 祝えよ天よ、天かける乙女たちよ\\
% いざなえ、我らを至福の館へと\\
% 我らみな手をたずさえて\\
% 全てのものが祝うのだ、この貴き日を\\
%
% 歌えよこの歓びを、この幸いを\\
% 人々も、神々も、動物も、虫たちも、\\
% 我らみな手をたずさえて\\
% 全てのものが祝うのだ、この貴き日を\\
%
% 飛び立て遠く彼方へ\\
% 衣よ、われらが愛しき乙女よ\\
% その旅立ちこそ我らの歓喜、我らの希望\\
%
% 我がともがらよ、祝えよ、歌えよ\\
% いざ声あげん、歓喜の声を\\
% 天江衣の生誕の歓びを\\

ころもの\\
おたんじょうび\\
たいへんめでたいので\\

せいいっぱい わたくしたち\\
いわいたいですわ\\

みんなのえがおで\\
こころつながる\\
とってもうれしい\\
Happy Birthday イェイ~\\

ランラランラランララン♪\\
ランラランラランララン♪\\
ランラランラランララン♪\\
ランラランラランララン♪\\

ころもは\\
いもうとみたい\\
ちいさくってかわいい\\

むちゅうで おしゃべりして\\
しあわせみつけた\\

みんなのえがおで\\
こころぽかぽか\\
ずっといっしょに\\
あそびましょイェイ~

\vfill\mincho
(未発売)
}


\addcontentsline{toc}{chapter}{\shingo 映画&ドラマ「咲-Saki-」}
\album{きみにワルツ}

\songTitled{きみにワルツ}{清澄高校麻雀部\\
宮永咲(浜辺美波)・原村和(\ruby{浅川梨奈}{SUPER☆GiRLS})・片岡優希(\ruby{廣田あいか}{私立恵比寿中学})・竹井久(古畑星夏)・染谷まこ(山田杏奈)}
{映画&ドラマ「咲-Saki-」オープニングテーマ}
\songMemo{作詞・作曲:\B{ユカ(みみめめMIMI)}/編曲:\B{CHRYSANTHEMUM BRIDGE}}
\songText{\kasho
きみにワルツ 繋ぐメロディー\\
星が眠る五線譜で ね 答えて\\

廻る地球の「真ん中」見当たらなくたって\\
「きみがいる場所」は いつだって探し出せるのは\\
ぼくの心 きみを中心に公転してるから\\

もしも 涙が溢れる夜が来たら\\
星を見上げてしまえば 願いに変わる\\

きみにワルツ 繋ぐメロディー\\
星が眠る五線譜に 続く未来は\\
きみがくれたの ひとりじゃないソラ\\

どんな高級な夢より\\
たった一瞬だって ね\\
その笑顔が 最高のギフト\\

くしゃくしゃな 愛の音符で\\
ありがとう 宙に舞って\\
クレッシェンドしてく明日へ\\
奏で 贈るから\\
ドレ ミラい ソラへ ぼくら踊ろうよ\\

ベンチ越しに見つけた その涙の雫まで\\
架かる虹をすぐ いつだって編み出せるのなら\\
棘々する すれ違いも 解くこと出来る?\\

もしも 願いが叶うなら 一秒一瞬\\
きみと同じ気持ちでいる\ruby{物語}{ストーリー}\\
\newcolumn
きみにワルツ 繋ぐメロディー\\
星が眠る五線譜に 続く未来は\\
きみがくれたの ひとりじゃないソラ\\

どんな高級な道より\\
きみと歩く砂利道に\\
咲く笑顔が 最高のギフト\\
勇気凛々 さあ 舞いあがれ\\[1em]

時に喧嘩して たまに傷ついて\\
絆でチューニング\\
どの 未来 空も 奇跡だったんだ\\[1em]

1!2!繋ぐメロディー\\
星が眠る五線譜に 続く未来は\\
きみがくれたの ひとりじゃないソラ\\

どんな高級な夢より\\
たった一瞬だって ね\\
咲く笑顔が 最高のギフト\\

くしゃくしゃな 愛の音符で\\
ありがとう 宙に舞って\\
クレッシェンドしてく明日へ\\
奏で 贈るから\\
ドレ ミラい ソラへ\\
ドレ ミラい ソラへ\\
ドレ ミラい ソラへ\\

きみと咲き誇れ
}

\songTitled{NO MORE CRY}{清澄高校麻雀部\\
宮永咲(浜辺美波)・原村和(\ruby{浅川梨奈}{SUPER☆GiRLS})・片岡優希(\ruby{廣田あいか}{私立恵比寿中学})・竹井久(古畑星夏)・染谷まこ(山田杏奈)}
{映画&ドラマ「咲-Saki-」エンディングテーマ}
\songMemo{作詞:\B{吉田安英}/作曲:\B{IKUMA}/編曲:\B{柳野裕孝(PRIMAGIC)}}
\songText{\kasho
NO MORE CRY NO MORE CRY\\
NO MORE CRY NO MORE CRY\\
明日へ NO MORE CRY\\
明日へ NO MORE CRY\\

こんなに大切な人が 近くにいることさえも\\
気づかずに 涙流したり\\
机の落書きの中に 輝いた夢のカケラ\\
ひとり胸に隠したり\\

そんな LONELY DAYS\\
繰り返すのはもう止めよう\\
この空の色に変えよう Yeah\\
逃げ出すことに慣れていたんだ\\
自分だけだとすさんでたんだ\\
そんな昨日の僕にはサヨナラ\\

走り出すよ\\NO MORE CRY NO MORE CRY\\
君の手を引いてゆける\\
明日へ NO MORE CRY NO MORE CRY\\
悲しみじゃなく 喜びの涙を流したい\\

あの日の君の優しさに 素直になれない僕は\\
小さな手を振り払った\\

ほどけたままの靴ヒモじゃ 走れないよ\\
この空を深く吸い込もう Yeah\\
大切なこと 忘れてたんだ\\
未来から目をそらしてたんだ\\
そんな昨日の僕にはサヨナラ\\
\newcolumn
走り出すよ\\NO MORE CRY NO MORE CRY\\
君がそばにいてくれた\\
笑顔で NO MORE CRY NO MORE CRY\\
何度でもやり直せる\\

Talk to me about you, baby \\
こんなにも君を求める\\
いつでも Stay with me for loving you, baby\\
ひとりでは生きてゆけない きっと誰もが\\
凍えた指先を 触れて温めたいよ\\
目に映る全てを今は 抱きしめたい\\

走り出すよ\\NO MORE CRY NO MORE CRY\\
君の手も引いてゆける\\
明日へ NO MORE CRY NO MORE CRY\\
悲しみじゃなく 喜びの涙を流したい\\

NO MORE CRY NO MORE CRY\\
悲しみじゃなく 喜びの涙を流したい
}



\end{document}
