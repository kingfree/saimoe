\chapter*{总序}
\addcontentsline{toc}{chapter}{总序}

最萌战争,浩浩汤汤,成王败寇,输酸赢婊。
\\

最萌大赛,也就是「萌战」。从2002年开办在2ch动画版的首届比赛开始,如今已经15年了。这15年间,无数大大小小的萌战承载了太多太多的恩怨,也见证着日本动画业乃至世界二次元亚文化的兴衰。「最萌十五年」也是新世代的十五年,是信息时代和动画业界的交合产物,更是这个产业文化和历史的忠实写照。

日萌、世萌、度萌、韩萌、300萌、新星萌、B萌……,萌战、燃战、联赛、动画鉴赏、万人投票……。数不胜数,林林总总。然而无论如何,日萌在萌战历史中享有的地位依然无可替代,上古时代日间动画制霸,后来厨团加入使得萌战人气速上,直到海外加入掀起萌战热潮,最后死于厨团把持,十三世而亡。期间世萌开办,规模巨大纷争繁多,高潮之后颓势尽显。直到2015年冬,哔哩哔哩举办B萌,因其参与人数众多,一跃而起成为王者。

我有幸从2012年入宅便见证园城寺怜登顶,更一手导致了2014年咲和双萌王旋踵而亡。日萌之死,表面上看是厨团打压散票所致,然而对照其他萌战,便知不止如此。萌战是基于网络的投票比赛,而近年来互联网技术的突飞猛进,则是摧毁日萌的重要一手。

由于互联网普及,大中学生可以参与刷票活动;由于终端设备繁多,一人的简单多设备多重也难以杜绝;由于投票机制不能与时俱进,陈旧的发行所和投票版程序被大厨解读得骨肉可辨;由于互联网金融体系完善,少许资金投入便可买到数量巨大的实体票源。这一切原因正造就了萌战当今现象——以日萌为代表的旧萌战体系逐渐瓦解,以B萌为代表的新萌战体系逐步建立。新旧体系的最重要区别在于参赛人数和投票数量的多少。依托中国浩瀚的人口和较高的教育程度,日系萌战被海外把持至死,国内萌战则是票数越来越多。票数的增加扩大了萌战军备的规模,厨团冲突更加激烈,各中小阵营连横合纵更为频繁。由于受众群体扩大,大量的散票更成为难以评定的不稳定因素,对新萌战产生着促进作用。

我们可以说,这是一个新旧交替的时代。它依然是萌战时代的一部分,只不过从初级阶段演进到了中级阶段。这不仅仅反映了萌战的重心转移,也反映了日本动画产业的重心转移——从日本资本和消费群体主导,到中国资本和消费群体主导。这将势必引起更多的连锁反应,而这浪潮我们无法阻挡!
\\

正如三年前偶然地卷入了萌战的旋涡,一发而不可收拾地结束了日萌。沉寂了两年左右没有参与萌战相关的事情,终于去年12月中旬在闹群的提议下,开始筹备闹萌。于此同时,也萌发了整理过去文章出本子的想法,当即便整理了军师的《咲-Saki-~萌战篇》和我的《2014日萌全程回顾》。后来因为工作原因,相关事情搁置了下来。直到2月份完成闹萌网站的开发以及3月份闹萌的运营,才又开始了这本小册子的编辑。

当初写的时候没有想到,这点内容足有8万余字,排印出来近200页,相当于一本书的容量了。加上时间过去了很久,不少图表已经佚失,不得不重新绘制和整理。也由于多年未从事排版工作,\LaTeX 重新拾起,花了不少时间。另外我还整理了《天才麻将少女》系列OP、ED和角色歌的歌词,附于全书最后,每页一首,便于读者唱诵和收藏。

本质上是自娱自乐性质的内部刊物,印刷出来也更多是纪念意义,所以对于内文内容没有做太多修改。除为了便于阅读加上的\uline{人名专名号}和\uwave{作品专名号}之外,适当调整了顺序和分段,以便适合纸质书籍的阅读。另外,凡是引用日语原文的,皆用日文字体排印。表格和图片直接附在文字之间,便于互相参考。

文本所述皆为陈年往事,恩怨纷争故年旧土,文华修辞不尽如意,读者权当娱乐即可。此书付梓,全无盈利目的,所有者尽可自由分发或拷贝,当然一把火付之如炬也没什么关系。如非必要,不会再版。是非对错,也交给历史评说。
\\

正值清明时节,日萌墓木已拱,不由感怀万分。沉舟侧翻,但想今年夏季的B萌,又当是一场百舸争流的最萌战争了罢!

\begin{flushright}
  \zihao{4}\rm\kasho 王者自由

  (咲衣憧)

  \kai 2017年4月4日
\end{flushright}
