% \Chapter{序}
%
% \begin{flushright}
%   \kai 偷鸡大师
%
%   \kai 2017年4月22日
% \end{flushright}

\Chapter{前言}

闹萌——东山奈央最萌大赛,是百度贴吧「东山奈央吧」吧务组,以及「奈央同萌会」群共同发起并举办的萌战活动。

起因是2016年11月17日,东山奈央宣布歌手出道,群里讨论如何庆祝。12月,千樱提出“不如举办个萌战”,我顿时眼前一亮,表示赞成。于是很快确定了“奈央萌”的名字——“闹萌”——取{\mincho 「奈央」}读音和昵称为“闹”,取“萌战”首字为“萌”。在比赛举行上,以东山出演所有动画、广播等角色参赛,不使用贴吧投票系统,自由负责投票网站的开发,另外请画师绘制宣传图和萌王图,初步定于2017年2月到3月期间举办——同时纪念2月1日首单发售和3月11日奈央生日。

这种没有任何实际回报、单纯凭借“爱”进行下去的活动,在一片“咕咕”声中开始了拖延大赛。小白首先收集了提名,讨论之后决定加入广播剧角色并去掉七山奈央,形成了158名的角色表;自由为网站购买了域名和VPS,简单写了一部分代码之后便放下了相关事务;KA制定了赛程和比赛规则;琉璃将提名人物用Excel分成8组。接近年末,大家又都忙了起来,直到农历新年过后,又才开始了相关准备。自由重新开始网站设计和开发,千樱收集了角色头像并联系了画师深蓝杰克绘制闹萌宣传图,小雪和鑫哥合作设计了网站图标。最后赛制一再调整,终于在2月21日释放出预赛分组,确定了最终赛程为2月23日〜3月11日。当然,手忙脚乱整理下来的结果就是,说好的“闹股”(萌股)变成“闹咕”了。

赛制很简单,预选赛8组4天进行,每组可投8位,前4名晋级本战。本战32强分为8组,两轮交叉进行,都是1v1。决赛8强分为4组,交叉三轮决出萌王。为了缓冲分组的压力,分组期间进行表演赛。比赛结束后,自由负责集计最终结果。运营组的根据个人时间撰写战报。到了比赛后半期,由于不正投票越来越多,运营也参与了对多重票的审议和处理。最终团子当选萌王,千樱邀请画师{\mincho ぴょん吉}绘制了萌王图,持续三周多的闹萌圆满结束。

我不禁想到了2014年的日萌,5位东山奈央配音的角色在四大阵营的纷争中全部晋级,中川花音、桐崎千棘、由比滨结衣季后赛止步32强,新子憧、九条可怜则决赛圈止步16强。东山阵营在麻厨、圆厨、美旋厨的纠葛与合力之下,达成了前所未有的佳绩,而两轮之间全部败退,昙花一现。这5个角色,也正是本届闹萌的4强。萌王由比滨结衣,准萌桐崎千棘,三位中川花音,四位新子憧,八强九条可怜。这样的巧合不禁叫人感慨,距离末代日萌已经过去了3年,最萌的东山角色还是没有变化。

如果说作为“声优厨”的闹厨为东山奈央在国内的人气做足了后援力量的话,那作为“声豚”的扭曲的闹厨则是推波助澜把“东山大法”宣扬到了几乎世人皆知的地步。2010年以中川花音出道的东山奈央,虽然尚显稚嫩却彰显实力。在2012年开办花音1st LIVE获得成功,又经过新子憧的发酵,在华语圈形成了相当大的跟风浪潮,火了起来。2013年,佐佐木千穂接连由比滨结衣两个属性相似的角色,给人留下了深刻的印象,并随着九条可怜以及金刚四傻系列,延伸出了更多的属性和萌点。2014年开办花音2nd LIVE并宣布花音毕业,同年的桐崎千棘为东山时代推上了高潮,并在2015年接到了东山狗泡芙这一长期饭票。2016年以女武神的身份再次洗脑中文圈,2017年终于迎来盼望已久的歌手出道。

最后,趁着闹萌的热劲儿,以及闹闹二单即将发售的机会,我们决定参展国内最大的同人展——上海ComiCup,为东山奈央再做宣传。于此,我整理本届闹萌历程发布本刊,为这一赛事留下一份纸质的记忆。

\begin{flushright}
  \kasho 王者自由(咲衣憧)

  \kai 2017年4月22日
\end{flushright}
